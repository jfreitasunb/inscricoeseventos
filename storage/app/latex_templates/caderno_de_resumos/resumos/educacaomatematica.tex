%====================================================================
%	Resumos de Educacao Matematica
%====================================================================

\pagestyle{fancy}

\lhead{}
\chead{{X Workshop de Ver\~{a}o em  Matem\'{a}tica - MAT/UnB}}
\rhead{}
%P\'{a}gina \arabic{page} de \pageref{ultimapagina}

\lfoot{Educa\c{c}\~{a}o Matem\'{a}tica}
\cfoot{\arabic{page}}
\rfoot{Ver\~{a}o \ano}

\renewcommand{\headrulewidth}{0.4pt}
\renewcommand{\footrulewidth}{0.4pt}

%====================================================================
%====================================================================

\begin{center}
	%\vspace{1cm}
	\huge{{\bf Educa\c{c}\~{a}o Matem\'{a}tica}}
	\vspace{1cm}
\end{center}

%====================================================================
%	Inicio dos Resumos de Educacao Matematica
%====================================================================


	\talktitle[Geraldo Eustáquio Moreira]{Tendências em educação matemática com enfoque na atualidade}
	
	\authorinfo[geust2007@gmail.com]{Geraldo Eustáquio Moreira}{Universidade de Brasília (FE)}
	
	\noindent\textbf{Resumo}.\index{Geraldo E. Moreira}\label{gem} 
	O aprimoramento nas formas de ensinar e aprender Matemática tem ganhado destaque nas últimas décadas, principalmente a partir de 1980. A formação do professor que ensina Matemática tem acompanhado, em certa medida, essa evolução, assentada, sobretudo, nas Tendências em Educação Matemática, mediante a força que imprime determinada orientação para a atuação docente, cujos impactos oriundos das transformações sociais, pedagógicas e tecnológicas, ditam o fazer pedagógico. Assim, falar em Tendências em Educação Matemática significa falar nos mecanismos sociais e pedagógicos que regulam as escolhas dos professores que ensinam Matemática, da Educação Infantil à Educação Superior. Então, minha fala objetiva identificar, contextualizar e analisar as principais Tendências em Educação Matemática, considerando o período atual, mas partindo das contribuições do início da solidificação da Educação Matemática. Intenciono, ainda, caminhar pela História da Educação Matemática no Brasil, situando o leitor acerca dos principais marcos teóricos relacionados ao surgimento de cada tendência abordada.  


\begin{thebibliography}{999}
	\bibitem{d1} BASSANEZI, Rodney C. Ensino-aprendizagem com modelagem matemática: uma nova estratégia. São Paulo: Contexto, 2002. 
	
	\bibitem{d2} BICUDO, Maria Aparecida V. (Org.). Pesquisa em educação matemática: concepções e perspectivas.São Paulo: Unesp, 1999.
	
	\bibitem{d3} BICUDO, Maria Aparecida Viggiani; GARNICA, Antonio Vicente Marafioti. Filosofia da Educação Matemática. 2. ed. Belo Horizonte: Autêntica, 2002.
	
	\bibitem{d4} BIEMBENGUT Maria Salett; HEIN, Nelson. Modelagem matemática no ensino. 2. ed. São Paulo: Contexto, 2002.
	
	\bibitem{d5} D’AMBRÓSIO, Ubiratan. Educação matemática: da teoria à prática. 4. ed. Campinas/SP: Papirus, 1998.
	
	\bibitem{d6} D’AMBRÓSIO, Ubiratan. Etnomatemática: elo entre as tradições e a modernidade. Belo Horizonte: Autêntica, 2001.
	
	\bibitem{d7} FIORENTINI, Dario. Alguns modos de ver e conceber o ensino da matemática no Brasil. Zetetiké, Campinas, n. 4, p. 1-37, nov. 1995.
	
	\bibitem{d8} MANRIQUE, Ana Lúcia; MARANHÃO, Maria Cristina Souza de Albuquerque; MOREIRA, Geraldo Eustáquio. Desafios da Educação Matemática Inclusiva: Formação de Professores. Volume I. São Paulo: Editora Livraria da Física, 2016.
	
	\bibitem{d9} MANRIQUE, Ana Lúcia; MARANHÃO, Maria Cristina Souza de Albuquerque; MOREIRA, Geraldo Eustáquio. Desafios da Educação Matemática Inclusiva: Práticas. Volume II. São Paulo: Editora Livraria da Física, 2016.
	
	\bibitem{d10} MOREIRA, Geraldo Eustáquio. O ensino de Matemática para alunos surdos: dentro e fora do texto em contexto. Educação Matemática Pesquisa, v. 18, p. 741-757, 2016. Disponível em: https://revistas.pucsp.br/index.php/emp/article/view/23486
	
	\bibitem{d11} MOREIRA, Geraldo Eustáquio. A Educação Matemática Inclusiva no contexto da Pátria Educadora e do novo PNE: Reflexões no âmbito do GD7. Educação Matemática Pesquisa (Online), v. 17, p. 508-519, 2015. Disponível em: https://revistas.pucsp.br/index.php/emp/article/view/25667
	
	\bibitem{d12} MOREIRA, Geraldo Eustáquio. Resolvendo problemas com alunos com Transtornos Globais do Desenvolvimento: desafios e conquistas. Educação Matemática em Revista-RS, v. 01, p. 38-48, 2014. Disponível em: http://sbemrs.org/revista/index.php/2011\_1/article/view/106
	
	\bibitem{d13} MOREIRA, Geraldo Eustáquio; MANRIQUE, A. L. Challenges in Inclusive Mathematics Education: Representations by Professionals Who Teach Mathematics to Students with Disabilities. Creative Education, v. 05, p. 470-483, 2014. Disponível em: http://file.scirp.org/Html/4-6302032\_45390.htm
	
	\bibitem{d14} ONUCHIC, Lourdes de la Rosa. Ensino-aprendizagem de matemática através da resolução de problemas. In: BICUDO, Maria Aparecida Viggiani (Org.). Pesquisa em educação matemática: concepções e perspectivas. São Paulo: Unesp, 1999.
	
	\bibitem{d15} POLYA, George. A arte de resolver problemas. Rio de Janeiro: Interciência, 2006.
	\end{thebibliography}

\vspace{24pt}
%====================================================================
	
	\talktitle[Janice Pereira Lopes]{O laboratório de educação matemática do IME/UFG: percurso histórico e novos desafios}
	
	\authorinfo[janice@ufg.br]{Janice Pereira Lopes}{Universidade Federal de Goiás}
	
	\noindent\textbf{Resumo}.\index{Janice P. Lopes}\label{scd} 
	Nas últimas décadas, o número de Laboratórios de Ensino vinculados aos cursos de Licenciatura em Matemática tem crescido. Na literatura, inclusive, é possível encontrar denominações distintas (Laboratório de Ensino; Laboratório de Educação Matemática, etc.) para espaços dessa natureza. Para Varizo (2007), “o que dá nome ao laboratório é a ciência objeto de seus estudos e experiências. No nosso caso [LEMAT], o objeto de estudos e experiências são as ciências da Educação voltadas para a Educação Matemática”. O Laboratório de Educação Matemática (LEMAT) do IME/UFG iniciou suas atividades em 1994, tendo como idealizadora e primeira coordenadora a professora Zaíra da Cunha Melo Varizo, que hoje dá nome ao LEMAT. O laboratório iniciou suas atividades e através de projetos submetidos a editais de financiamento do MEC e da Pró-Reitoria de Extensão da UFG para constituir seu acervo de materiais e equipamentos. Em seus mais de 20 anos de existência, o LEMAT tem exercido um papel crucial na formação inicial e continuada de professores de matemática, da UFG e de outras IES da região, se estabelecendo num espaço de referência para o desenvolvimento da Educação Matemática no Estado de Goiás. Por meio de distintos projetos de ensino e de pesquisa, o laboratório contribui significativamente para a qualidade da formação inicial do professor de matemática; além de ser lócus de produção acadêmica na área de Educação Matemática, de socialização de pesquisas, experiências e recursos diversos.
	

\begin{thebibliography}{999}
	\bibitem{} LORENZATO, S. Laboratório de Ensino de Matemática na formação de professores. Campinas: Autores Associados, 2012.
	
	\bibitem{} RÊGO, R.M.; RÊGO, R.G. Desenvolvimento e uso de materiais didáticos no ensino de matemática. In: LORENZATO, S. Laboratório de Ensino de Matemática na formação de professores. Campinas: Autores Associados, 2006, p. 39-56.
	
	\bibitem{} VARIZO, Z. C. M. O. Laboratório de Educação Matemática do IME/UFG: do sonho à realidade. In: IX Encontro Nacional de Educação Matemática - ENEM -, Belo Horizonte – MG, 18 a 21 de Julho, 2007, Anais.... Disponível em: http://www.sbembrasil.org.br/files/ix\_enem/index.htm. Acesso em: 25 jan. 2018.
	
	\bibitem{} VARIZO, Z. da C. M.; CIVARDI, J. A.. (Org.). Olhares e reflexões acerca de concepções e práticas no Laboratório de Educação Matemática. Curitiba: Editora CRV, 2011.
	\end{thebibliography}	
	
	\vspace{24pt}
%====================================================================

	\talktitle[Jorge Cássio C. Nóbriga]{Demonstrações matemáticas dinâmicas}

	\authorinfo[j.cassio@ufsc.br]{Jorge Cássio C. Nóbriga}{Universidade Federal de Santa Catarina}

	\noindent\textbf{Resumo}.\index{Jorge C. C. Nóbriga}\label{jccn} 
	Nesta palestra, apresentarei um novo conceito que tenho chamado de Demonstrações Matemáticas Dinâmicas. Não se trata de uma nova forma de demonstração, mas, sim, de como apresentá-la. Mais do que validar, o objetivo é explicar para que o estudante compreenda de fato. A criação de um novo conceito era necessária porque as demonstrações dinâmicas têm características próprias que diferem das demonstrações convencionais. Esse tipo de demonstração só é possível em ambientes de Geometria Dinâmica. Os primeiros experimentos feitos com estudantes de licenciatura em Matemática indicam que o uso das Demonstrações Matemáticas Dinâmicas pode auxiliar a compreensão, desenvolver a aprendizagem com autonomia e potencializar o poder argumentativo. Durante a palestra, mostrarei exemplos de Demonstrações Matemáticas Dinâmicas produzidas nas plataformas GGBOOK e GeoGebra.


\begin{thebibliography}{999}
	\bibitem{c1} BICUDO, I. Demonstração em Matemática. BOLEMA: Boletim de Educação Matemática. Rio Claro: Editora Unesp. Ano 15, n.18, pp. 79-90, set. 2002.
	
	\bibitem{c2} DUVAL, R. Registros de representações semióticas e funcionamento cognitivo da compreensão em matemática. In: MACHADO, S. D. A. Aprendizagem em matemática: registros de representação semiótica. Campinas: Papirus, p. 11–33, 2008. 
	
	\bibitem{c3} NÓBRIGA, J. C. C.; LACERDA SANTOS, G.; ARAÚJO, L. C. L.; FERREIRA, B. S.; LIMA. GGBook: One interface wich integrates the text and graphic environments in the Geogebra. In: 12 TH INTERNATIONAL CONGRESS ON MATHEMATICAL EDUCATION. Anais… Seul: 2012a Disponível em: <http://www.academia.edu/6633172/GGBook\_One\_interface\_wich\_integrates\_the\_text\\ \_and\_graphic\_environments\_in\_the\_Geogebra\_Anais\_do\_ICME\_2012\_>. Acesso em: 26 janeiro. 2018
	
	\bibitem{c4} NÓBRIGA, J. C. C. ; LACERDA SANTOS, G.; ARAÚJO, L. C. L.; FERREIRA, B. S.; LIMA. GGBOOK: uma interface que integrará os ambientes de texto e gráfico no GeoGebra. Revista do Instituto GeoGebra Internacional de São Paulo. ISSN 2237-9657, v. 1, n. 1, p. 03 – 12, 12 mar. 2012b. 
	
	\bibitem{c5} NÓBRIGA, J. C. C. Aprendendo Geometria Plana com a Plataforma GeoGebra.  Disponível em: < https://www.geogebra.org/m/hsXHDRX7> \_>. Acesso em: 26 jan. 2018
\end{thebibliography}

\vspace{24pt}

%====================================================================

	\talktitle[Sérgio Carrazedo Dantas]{Resolvendo problemas com o Geogebra}

	\authorinfo[sergio.dantas@unespar.edu.br]{Sérgio Carrazedo Dantas}{Universidade Estadual do Paraná}

	\noindent\textbf{Resumo}.\index{Sérgio C. Dantas}\label{scd} 
	Nessa palestra, apresento minha perspectiva pessoal de resolução de problemas utilizando o GeoGebra. Abordo problemas de aritmética, álgebra e geometria, explorando algumas formas de resolução a partir do que é proposto em seus enunciados. Além disso, proponho investigações suscitadas pela leitura dos enunciados que vão além de suas proposições iniciais, ou seja, proponho uma forma de utilização do GeoGebra para desdobramento de problemas em outros mais complexos e generalizadores, em um trabalho que se assemelha ao que é realizado com a metodologia de investigação matemática.
	
	
	\begin{thebibliography}{999}
	\bibitem{} BARANAUSKAS, M. C. C.; MARTINS, M. C.; VALENTE, J. A. Codesign de redes digitais: tecnologias e educação a serviço da inclusão social. Porto Alegre: Penso, 2013. 
	
	\bibitem{} BARRABÁSI, A.-L. Linked: A nova ciência dos networks. Tradução de Jonas Pereira dos Santos. São Paulo: Leopardo Editora, 2009.
	
	\bibitem{} DANTAS, S. C. Pressupostos para Formação de Professores de Matemática em um Curso via Web. Revista Perspectivas da Educação Matemática, Campo Grande, v. 8, n. 16, p. 308-331, 2015.
	
	\bibitem{} LEONTIEV, A. O desenvolvimento do psiquismo. Lisboa: Horizonte Universitário, 1978.
	
	\bibitem{} LINS, R. C. Por que discutir teoria do conhecimento é relevante para a Educação Matemática. In: BICUDO, M. A. V. (ORG.). Persquisa em Educação Matemática: Concepções \& Perspectivas. São Paulo: Editora UNESP, 1999. Cap. 4, p. 75-94.
	
	\bibitem{} PINTO, Á. V. O conceito de tecnologia. Rio de Janeiro: Contraponto, v. I e II, 2005.
	\end{thebibliography}

\vspace{24pt}
%====================================================================

	\talktitle[Wellington Lima Cedro]{Licenciatura em matemática: reflexões sobre o ontem, o hoje e o amanhã}
	
	\authorinfo[wcedro@ufg.br]{Wellington Lima Cedro}{Universidade Federal de Goiás }
	
	\noindent\textbf{Resumo}.\index{Wellington L. Cedro}\label{wlc} 
	A partir da promulgação de uma série de diretrizes legais pelo Conselho Nacional de Educação, a formação docente passa a ser compreendida como um processo permanente que envolve tanto a valorização da identidade do professor como da sua profissionalidade. No ano de 2015, com a publicação das “Diretrizes Curriculares Nacionais para a Formação Inicial em nível superior (cursos de licenciatura, cursos de formação pedagógica para graduados e cursos de segunda licenciatura) e para a formação continuada” (Resolução CNE/CP nº 2/2015), surge a necessidade, mais uma vez, e a oportunidade de repensarmos os caminhos dos cursos de Matemática. Assim, este é mais um momento em que precisamos fazer uma análise aprofundada dessas diretrizes, para que se possa pensar em quais são os impactos desta resolução para os cursos de licenciatura. Com esse objetivo, faremos, nesta apresentação, uma breve análise histórica da licenciatura em nosso país e apontaremos os desafios que temos pela frente com o advindo desta nova diretriz.
	
	
	\vspace{24pt}
%====================================================================

	\talktitle[Wesley Pereira da Silva]{A Tecnologia Assistiva para Estudantes com Defici\^{e}ncia Visual}

	\authorinfo[wesleynh3@gmail.com]{Wesley Pereira da Silva}{Secretaria de Estado de Educação do Distrito Federal}

	\noindent\textbf{Resumo}.\index{Wesley P. Silva}\label{wps} 
	A tecnologia assistiva é uma excelente ferramenta para o ensino do estudante com deficiência visual. O Código Braille, o soroban e o computador são exemplos de tecnologia assistiva que auxiliam o estudante com deficiência visual no processo de inclusão escolar. No cotidiano escolar do estudante com deficiência visual, é importante o uso de recursos lúdicos, para isso, contamos com diversos recursos disponíveis no Sistema Dosvox. Tal sistema foi criado pelo Núcleo de Computação Eletrônica da UFRJ e consiste em um sistema integrado com mais de noventa programas que apresenta, de forma diferenciada, as funções de um computador para a pessoa com deficiência visual. Na área educacional, o Sistema Dosvox proporciona o aprendizado de conceitos de forma lúdica por meio dos jogos digitais que estão disponíveis. O programa Jogavox permite a criação de jogos digitais adaptados para a pessoa com deficiência visual de forma simplificada. Tal ferramenta admite a inserção de uma gama de recursos audiovisuais nos jogos digitais adaptados. Os recursos audiovisuais permitem que os estudantes videntes também utilizem os jogos digitais adaptados, proporcionando à sala de aula um ambiente de inclusão.
	
	\begin{thebibliography}{999}
	\bibitem{} BORGES, José Antonio dos Santos; PAIXÃO, Berta Regina; BORGES, Sônia. Alfabetização de crianças cegas com ajuda do computador. Disponível em: <intervox.nce.ufrj.br/dosvox/textos/dedinho.doc>. Acesso em: 10 fev. 2018.

	\bibitem{} BORGES, José Antonio dos Santos. Do Braille ao Dosvox – diferenças nas vidas dos cegos brasileiros – Rio de Janeiro: UFRJ/COPPE, 2009. 2009. 327 f. Tese (doutorado) – UFRJ/ COPPE/ Programa de Engenharia de Sistemas e Computação, 2009.
	
    \bibitem{} BUENO, Salvador Toro; MARTÍN, Manuel Bueno. Deficiência Visual: aspectos pisicoevolutivos e educacionais. Trad. Magali de Lourdes Pedro. São Paulo: Santos, 336 p. 2010.

	\bibitem{} DISTRITO FEDERAL.  Secretaria de Estado de Educação. Orientação Pedagógica: Educação Especial. Brasília, 2010. Disponível em: http://www.cre.se.df.gov.br/ascom/documentos/subeb/ed\_especial/orient\_pedag\_ed\_especial2010.pdf. Acesso em: 10 fev. 2018.

	\bibitem{} RAPOSO, Patrícia Neves; MÓL, Gerson de Souza. A Diversidade para Aprender Conceitos Científicos: a ressignificação do Ensino de Ciências a partir do trabalho pedagógico com alunos cegos. In: SANTOS, Wildson Luiz Pereira dos; MALDANER, Otavio Aloisio (Orgs.). Ensino de Química em foco, Ijuí: Ed. Unijuí, 2011.

	\bibitem{} SCHWARZ, A.; HARBER, J. População com deficiência no Brasil: fatos e percepções. São Paulo: Febraban, 2006. 42p. Disponível em: <http://www.febraban.org.br/Arquivo/Cartilha/Livro\_Popula\%E7ao\_Deficiencia\_Brasil.pdf>. Acesso em: 10 fev. 2018.

	\bibitem{} VIGOTSKI, Liev Semionovitch. A brincadeira e o seu papel no desenvolvimento psíquico da criança. Revista Digital de Gestão de Iniciativas Sociais. Rio de Janeiro, v. 5, n. 11, p. 23-36, jun. 2008. ISSN 1808-6535

	\bibitem{} VIGOTSKI, Liev Semionovitch. A formação social da mente: o desenvolvimento dos processos psicológicos superiores. Trad. José Cipolla Neto, Luís Silveira Menna Barreto, Solange Castro Afeche. 6ª ed. São Paulo: Martins Fontes, 1998.

	\end{thebibliography}

\vspace{24pt}
%====================================================================
%	Fim dos Resumos de Educacao Matematica
%====================================================================	


\clearpage	