%====================================================================
%	Resumos de Mini-Cursos
%====================================================================

\pagestyle{fancy}

\lhead{}
\chead{{X Workshop de Ver\~{a}o em  Matem\'{a}tica - MAT/UnB}}
\rhead{}
%P\'{a}gina \arabic{page} de \pageref{ultimapagina}

\lfoot{Mini-Cursos}
\cfoot{\arabic{page}}
\rfoot{Ver\~{a}o \ano}

\renewcommand{\headrulewidth}{0.4pt}
\renewcommand{\footrulewidth}{0.4pt}

%====================================================================
%====================================================================

\begin{center}
	%\vspace{1cm}
	\huge{{\bf Mini-Cursos}}
	\vspace{1cm}
\end{center}

%====================================================================
%	Inicio dos Resumos de Mini-Cursos
%====================================================================

	\talktitle[Andre Caldas de Souza, Lucas Seco Ferreira, Mauro Patrão]{Dinâmica, geometria e simetria}
	
	\authorinfo[andrecaldas@unb.br]{Andre Caldas de Souza}{Universidade de Bras\'{i}lia}
	
	\authorinfo[lseco@unb.br]{Lucas Seco Ferreira}{Universidade de Bras\'{i}lia}
	
	\authorinfo[m.m.a.patrao@mat.unb.br]{Mauro Patrão}{Universidade de Bras\'{i}lia}
	
	\noindent\textbf{Resumo}.\index{Andre C. Souza}\index{Lucas S. Ferreira}\index{Mauro Patrão}\label{acs}\label{lsf}\label{mp} 
	Nesse minicurso, vamos apresentar um panorama geral e acessível de assuntos relacionadas à pesquisa do grupo de Sistemas Dinâmicos da UnB. Vamos focar em exemplos interessantes e apresentar as partes mais simples das técnicas e demonstrações.
	
	Nas primeiras três aulas, vamos falar sobre dois tipos de sistemas dinâmicos onde as simetrias aparecem de maneira natural. O primeiro tipo é a dinâmica de translações em variedades flag, bem exemplificado pela dinâmica de uma matriz agindo em direções no espaço projetivo. Conseguimos dar uma caracterização completa dos comportamentos transiente e recorrente. Isso era conhecido apenas no caso de uma matriz diagonal ou conforme. O segundo tipo é a dinâmica de um endomorfismo de um grupo de Lie, bem exemplificado pela aplicação elevar ao quadrado nos complexos menos a origem. Conseguimos dar uma caracterização completa da entropia topológica, dada pelo princípio variacional. Isso era conhecido apenas no caso de grupos compactos. Para isso, apresentaremos na aula anterior a entropia topológica e seu princípio variacional, o que era conhecido apenas para espaços compactos.
	Nas duas últimas aulas, vamos falar sobre a geometria de espaços com bastante simetria. Na quarta aula, após relembrarmos a noção de métrica invariante pela ação de um grupo de Lie, descrevemos todas as métricas invariantes em variedades flag de formas reais normais, o que era conhecido apenas para variedades flags complexas. Na última aula, vamos mostrar como contar as geodésicas ligando dois pontos de um grupo de Lie compacto com uma métrica biinvariante, o que era conhecido apenas para as geodésicas minimizantes de grupos simplismente conexos.
	
	\vspace{24pt}
%====================================================================
	
	\talktitle[Brigitte Lutz-Westphal]{Flexible thinking in the mathematics classroom}
		
	\authorinfo[email@unb.br]{Brigitte Lutz-Westphal}{Freie Universität Berlin}
		
	\noindent\textbf{Resumo}.\index{Brigitte Lutz-Westphal}\label{blw} 
	Flexible thinking is an important element in mathematics education. It allows to grasp mathematical phenomena more clearly (cf. Wittmann (1985)) and trains the questioning of given facts. For a research-based approach to mathematics (cf. Ludwig, Lutz-Westphal, Ulm, 2017), flexible thinking is essential. It's also fun to think "around the corner" this way. This workshop introduces flexible thinking tasks (Lutz-Westphal (2018)) that can be applied to all levels of education. Subsequently, such tasks are jointly developed and tested in the group, so that the participants take home specific ideas for mathematics classroom activities.
		
	\selectlanguage{english}
		
	\begin{thebibliography}{999}
	\bibitem{b1} Ludwig, Matthias; Lutz-Westphal, Brigitte; Ulm, Volker (2017): Forschendes Lernen im Mathematikunterricht. Mathematische Phänomene aktiv hinterfragen und erforschen. In: Praxis der Mathematik, 73/59, p. 2-9.
	
	\bibitem{b2} Linke, Pauline; Lutz-Westphal, Brigitte (2018): Das „Spot-Modell“ im Mathematikunterricht - forschendes und entdeckendes Lernen fundiert anwenden, to appear.
	
	\bibitem{b3} Wittman, Erich Ch. (1985): Objekte - Operationen - Wirkungen: Das operative Prinzip in der Mathematikdidaktik. In: Mathematik lehren 11, p. 7-11
	\end{thebibliography}
	
		\vspace{24pt}
	
%====================================================================

	\talktitle[Hugo Tadashi Muniz Kussaba]{\href{http://www.mat.unb.br/verao/Uploads/controle_sd.pdf}{Introdução à teoria de controle de sistemas dinâmicos}}
	
	\authorinfo[kussaba@lara.unb.br]{Hugo Tadashi Muniz Kussaba}{Universidade de Bras\'{i}lia}
	
	\noindent\textbf{Resumo}.\index{Hugo T. M. Kussaba}\label{htmk} 
	A teoria de controle é uma área da matemática aplicada cujo foco é a análise e o projeto dos chamados sistemas de controle, isto é, sistemas que permitem a atuação de uma entrada no sistema que pode influenciar o comportamento do mesmo. A pergunta básica da teoria de controle é como projetar esta entrada de modo a se obter um comportamento desejado. Devido à sua versatilidade, a teoria de controle encontra aplicações desde em problemas de engenharia, como robótica, até problemas de economia e ecologia.
	
	Neste mini-curso abordaremos uma introdução à teoria de controle de sistemas dinâmicos lineares, tanto de sistemas em tempo contínuo quanto de sistemas em tempo discreto, com maior enfoque no caso de tempo contínuo. Serão tratados os conceitos de controlabilidade e observabilidade de sistemas dinâmicos lineares, critérios para analisar a estabilidade desses sistemas e como realizar a estabilização desses sistemas, assim como estimar o estado desses sistemas a partir de medições.
	
	\vspace{24pt}
%====================================================================

	\talktitle[Hiuri Reis]{O fluxo redutor de curvas e superfícies}
	
	\authorinfo[hiurifellipe@gmail.com]{Hiuri Reis}{Instituto Federal de Goiás}
	
	\noindent\textbf{Resumo}.\index{Hiuri Reis}\label{hr} 
	Neste minicurso, vamos discutir alguns aspectos da evolução de curvas pelo fluxo redutor de curvas. Vamos começar apresentando a definição de fluxo redutor de curvas no plano Euclidiano e discutiremos algumas de suas propriedades. Em seguida, vamos estudar o fluxo redutor de curvas em superfícies, discutindo suas principais propriedades. Vamos apresentar as soluções autossimilares do fluxo redutor de curvas no plano Euclidiano, dando a classificação e descrição destas curvas. Para finalizar, vamos definir as soluções sólitons do fluxo redutor de curvas em superfícies e apresentar alguns propriedades e exemplos destas curvas.
	
	\vspace{24pt}
%====================================================================
	
	\talktitle[Janice Pereira Lopes]{O laboratório de educação matemática do IME/UFG: percurso histórico e novos desafios}
	
	\authorinfo[janice@ufg.br]{Janice Pereira Lopes}{Universidade Federal de Goiás}
	
	\noindent\textbf{Resumo}.\index{Janice P. Lopes}\label{scd} 
	O crescente número de Laboratórios de Ensino que têm se constituído junto aos cursos de Licenciatura em Matemática sugere a importância que tais laboratórios representam no processo de formação inicial de professores de Matemática. Tal crescimento impulsiona reflexões e pesquisas em torno das características, objetivos e fundamentos destes laboratórios, resultando, inclusive, no surgimento de diferentes nomenclaturas e concepções. Laboratório de Ensino; Laboratório de Ensino e Aprendizagem, Laboratório de Educação Matemática, entre outros, são algumas das denominações utilizadas para espaços dessa natureza. Mais do que modos diferentes de denominação, ao avaliar mais cuidadosamente cada uma destas nomenclaturas, é possível perceber que, em alguns casos, elas carregam concepções distintas não só do papel do laboratório, mas, também, acerca da formação priorizada. No cenário nacional, desde a promulgação da LNDBE, em 1996, e posteriormente da Resolução CNE/CP nº2, de 2002, que determina 400 horas de estágio curricular supervisionado, a importância dos LEM para a formação do professor de matemática tem sido destacada. Na UFG, a resolução 332 do CCEP (UFG), de 1992, estabeleceu um novo currículo para o Curso de Matemática – Bacharelado e Licenciatura, e também determinou a criação de um Laboratório de Ensino de Matemática destinado, entre outras coisas, à análise e elaboração de materiais didáticos de Matemática a serem utilizados no Estágio Curricular Supervisionado. Esse minicurso pretende resgatar vestígios históricos do surgimento dos LEM, bem como algumas concepções e desafios que cercam a constituição e manutenção destes espaços. 
	
	
	\begin{thebibliography}{999}
		\bibitem{} LORENZATO, S. Laboratório de Ensino de Matemática na formação de professores. Campinas: Autores Associados, 2012.
		
		\bibitem{} RÊGO, R.M.; RÊGO, R.G. Desenvolvimento e uso de materiais didáticos no ensino de matemática. In: LORENZATO, S. Laboratório de Ensino de Matemática na formação de professores. Campinas: Autores Associados, 2006, p. 39-56.
		
		\bibitem{} VARIZO, Z. C. M. O. Laboratório de Educação Matemática do IME/UFG: do sonho à realidade. In: IX Encontro Nacional de Educação Matemática - ENEM -, Belo Horizonte – MG, 18 a 21 de Julho, 2007, Anais.... Disponível em: http://www.sbembrasil.org.br/files/ix\_enem/index.htm. Acesso em: 25 jan. 2018.
		
		\bibitem{} VARIZO, Z. da C. M.; CIVARDI, J. A.. (Org.). Olhares e reflexões acerca de concepções e práticas no Laboratório de Educação Matemática. Curitiba: Editora CRV, 2011. 
	\end{thebibliography}
		
		\vspace{24pt}
%====================================================================
	\talktitle[Kostiantyn Iusenko]{\href{http://www.mat.unb.br/verao/Uploads/notas\_algebra\_quivers.pdf}{Algebras, quivers and adjoint functors}}

	\authorinfo[iusenko@ime.usp.br]{Kostiantyn Iusenko}{Universidade de S\~{a}o Paulo (IME-USP)}
	
	
	\noindent\textbf{Resumo}.\index{Autor}\label{Autor} 
	The aim of this course is a primer with basic notions of category theory and representation theory of finite dimensional algebras. We will study the concept of adjoint functors and show that the correspondence “quiver” $\rightleftarrows$ “algebra” can be interpreted as a pair of adjoint functors between certain categories. Mostly the lectures do not contain the proofs and the theory is accompanied by examples (sometimes introduced as the exercises).  
	%	This mini-course is a short introduction to the basic concepts of category theory and representation theory of finite-dimensional algebras. We will learn the concept of adjoint functors and will show that the construction ''quiver'' <--> "algebra" can be interpreted as a pair of adjoint functors between certain categories. Lectures almost do not contain the proofs, theoretical part will be accompanied with examples, and sometimes introduced in the form of exercises.
	
	\vspace{24pt}
%====================================================================


	\talktitle[Marcos Tadeu Oliveira Pimenta]{\href{http://www.mat.unb.br/verao/Uploads/notas_problemas_quasilineares.pdf}{Problemas quasilineares elípticos modelados no espaço BV}}
	
	\authorinfo[mtopimenta@gmail.com]{Marcos Tadeu Oliveira Pimenta}{Universidade Estadual Paulista}
	
	\noindent\textbf{Resumo}.\index{Marcos T. O. Pimenta}\label{mtop} 
	Nesse minicurso definiremos e exploraremos as propriedades básicas do espaço das funções de variação limitada. Especial atenção será dada às aplicações a problemas elípticos que se modelam naturalmente nesse espaço.
	
	\vspace{24pt}
%====================================================================
	
	\talktitle[Michael Anton Hoegele]{A short introduction to Calculus with Lévy processes}
	
	\authorinfo[ma.hoegele@uniandes.edu.co]{Michael Anton Hoegele}{Universidad de los Andes}
	
	\noindent\textbf{Resumo}.\index{Michael A. Hoegele}\label{mah} 
	In this introductory talk we will introduce the Lévy processes and their representations and solutions of stochastic differential equations with Lévy noise and and their basic properties. 
	
	\vspace{24pt}
%====================================================================

	\talktitle[Scott CooK]{Random and No-Slip Billiard Dynamical Systems}
	
	\authorinfo[scook@tarleton.edu]{Scott CooK}{Tarleton State University}
	
	\noindent\textbf{Resumo}.\index{Scott CooK}\label{sc} 
	This workshop will explore two variants of billiard dynamical systems - no-slip billiards and random billiards. We will emphasize both the important mathematical definitions/results and modern computer-based simulation techniques using Python and its scientific stack.
	A no-slip billiard systems (introduced by Broomhead \& Gutkin [93]) uses a deterministic collision law for hard spheres with a single point of contact which allows transfer of momentum between translational and angular modes while conserving total kinetic energy. It act like a frictional force without complex deformations of the colliding spheres.  Students will simulate these no-slip system to explore how they contrast to standard specular systems.
	In a random billiard systems, the standard specular reflection law is given a random component via surface “microstructure”. These systems exhibit many of the properties of standard billiards system, but allow for scattering seen in real gas-surface interactions. We can define a notion of temperature for the surface, allowing a theory of stochastic thermodynamics (heat flow, entropy, work, etc) rooted in billiard dynamics. If time allows, we may discuss many particle systems which combine both of the collisions laws.
	
	\vspace{24pt}

%====================================================================
	
	\talktitle[Sérgio Carrazedo Dantas]{Sequências no Geogebra}
	
	\authorinfo[sergio.dantas@unespar.edu.br]{Sérgio Carrazedo Dantas}{Universidade Estadual do Paraná}
	
	\noindent\textbf{Resumo}.\index{Sérgio C. Dantas}\label{scd} 
	O presente minicurso tem por objetivo promover a integração de conhecimentos técnicos relativos ao software GeoGebra e conhecimentos matemáticos sobre sequências algébricas e numéricas. Nas atividades desenvolvidas, exploramos ferramentas e comandos do GeoGebra que permitem construir sequências e arranjos geométricos por meio de interações numéricas, isometrias dinâmicas e combinações de comandos internos do software GeoGebra. A participação nesse minicurso exige conhecimentos básicos sobre o GeoGebra.
	
	\begin{thebibliography}{999}
		\bibitem{} BARANAUSKAS, M. C. C.; MARTINS, M. C.; VALENTE, J. A. Codesign de redes digitais: tecnologias e educação a serviço da inclusão social. Porto Alegre: Penso, 2013. 
		
		\bibitem{} BARRABÁSI, A.-L. Linked: A nova ciência dos networks. Tradução de Jonas Pereira dos Santos. São Paulo: Leopardo Editora, 2009.
		
		\bibitem{} DANTAS, S. C. Pressupostos para Formação de Professores de Matemática em um Curso via Web. Revista Perspectivas da Educação Matemática, Campo Grande, v. 8, n. 16, p. 308-331, 2015.
		
		\bibitem{} LEONTIEV, A. O desenvolvimento do psiquismo. Lisboa: Horizonte Universitário, 1978.
		
		\bibitem{} LINS, R. C. Por que discutir teoria do conhecimento é relevante para a Educação Matemática. In: BICUDO, M. A. V. (ORG.). Persquisa em Educação Matemática: Concepções \& Perspectivas. São Paulo: Editora UNESP, 1999. Cap. 4, p. 75-94.
		
		\bibitem{} PINTO, Á. V. O conceito de tecnologia. Rio de Janeiro: Contraponto, v. I e II, 2005.
	\end{thebibliography}
	\vspace{24pt}
	
%====================================================================
%	Fim dos Resumos de Mini-Cursos
%====================================================================	


\clearpage	