%====================================================================
%	Resumos de Analise
%====================================================================

\pagestyle{fancy}

\lhead{}
\chead{{X Workshop de Ver\~{a}o em  Matem\'{a}tica - MAT/UnB}}
\rhead{}
%P\'{a}gina \arabic{page} de \pageref{ultimapagina}

\lfoot{An\'{a}lise}
\cfoot{\arabic{page}}
\rfoot{Ver\~{a}o \ano}

\renewcommand{\headrulewidth}{0.4pt}
\renewcommand{\footrulewidth}{0.4pt}

%====================================================================
%====================================================================

\begin{center}
	%\vspace{1cm}
	\huge{{\bf An\'{a}lise}}
	\vspace{1cm}
\end{center}

%====================================================================
%	Inicio dos Resumos de Analise
%====================================================================


	\talktitle[Ad\'{a}n J. Corcho]{On the blow-up phenomena for short an
	long waves interactions}

	\authorinfo[adan@im.ufrj.br]{Ad\'{a}n J. Corcho}{Universidade Federal do Rio de Janeiro}

	\noindent\textbf{Abstract}.\index{Ad\'{a}n J. Corcho}\label{adan} We will show that for the Schrödinger-Korteweg-de Vries system the formation of singularities appears for "focusing" interactions of the nonlinearities. The results are surprising since that both nonlinearities are subcritical for classical nonlinear Schrödinger and KdV equations.

\vspace{24pt}
%====================================================================

	\talktitle[Claudianor Alves]{Existência de solução para uma classe de problemas envolvendo o Laplaciano Fracionário em $\mathbb{R}^N$ via Teoria da Bifurcação.}
	
	\authorinfo[coalvesbr@yahoo.com.br]{Claudianor Alves}{Universidade Federal de Campina Grande}
	
	\noindent\textbf{Resumo}.\index{Claudianor Alves}\label{ca} 
	Nesta palestra iremos apresentar alguns resultados recentes envolvendo a existência de solução para uma classe de problemas envolvendo o Laplaciano fracionário em todo $R^N$. As principais ferramentas usadas são: O grau de Leray-Schauder, O índice de Leray-Schauder, O Teorema de Crandal -Rabinowitz e o Teorema Global de Bifurcação devido a Rabinowitz.

	\vspace*{0.5cm} \noindent Este é um trabalho em conjunto com os Professores Alânnio Nóbrega e Romildo  de Lima da UFCG.
	
	\vspace{24pt}
%====================================================================
	\talktitle[Claudio A. Gallegos]{Fixed points of multivalued maps under local Lipschitz conditions}

	\authorinfo[claudio.gallegos@usach.cl]{Claudio A. Gallegos}{University of Santiago}

	\noindent\textbf{Resumo}.\index{Claudio A. Gallegos}\label{cag} 
	We are concerned with the  existence of fixed points for multivalued maps defined on Banach spaces.
	Using the Banach spaces scale concept, we  establish the existence of a fixed point of a multivalued map in a vector subspace where the map is locally Lipschitz continuous. We apply our results to the existence of mild solutions and asymptotically almost periodic solutions of an abstract Cauchy problem governed by a first order  differential inclusion.

\vspace*{0.5cm} \noindent Joint work with Hern\'an R. Henr\'iquez.
	
\selectlanguage{english}

\begin{thebibliography}{999}
	\bibitem{HHGA} Gallegos~C. A. , Henr\'iquez~H. R. ,{\it Fixed points of multivalued maps under local Lipschitz conditions and applications}. (Submitted)
\end{thebibliography}

	\vspace{24pt}

%====================================================================
	\talktitle[Everaldo de Mello Bonotto]{Monotone impulsive dynamical systems}
	
	\authorinfo[ebonotto@icmc.usp.br]{Everaldo de Mello Bonotto}{Universidade de S\~ao Paulo (ICMC -- USP)}
	
	\noindent\textbf{Resumo}.\index{Everaldo M. Bonotto}\label{emb} 
	This talk is concerned with the theory of impulsive dynamical systems.
	We exhibit sufficient conditions for a set to be Zhukovskij quasi stable in dissipative monotone impulsive systems. Also, some recursive properties as minimality and recurrence are related
	with monotone impulsive systems.
	
	\selectlanguage{english}
	
	\begin{thebibliography}{999}
		\bibitem{Bonotto} E. M. Bonotto, Monotone impulsive dynamical systems. Collect. Math., (2018), 17-24.
	\end{thebibliography}
	
	\vspace{24pt}
%====================================================================
	\talktitle[Gaetano Siciliano]{On the Schr\"{o}dinger-Bopp-Podolsky system}
	
	\authorinfo[gaetano.siciliano@gmail.com]{Gaetano Siciliano}{Universidade de São Paulo (IME - USP)}
	
	\noindent\textbf{Resumo}.\index{Gaetano Siciliano}\label{gs} 
	We consider a new model describing the  interaction between matter and electromagnetic field: indeed the matter is described by the Schrödinger Lagrangian and the electromagnetic field by a  Lagrangian introduced by B. Podolsky in 1942, which is more refined with respect to the classical Maxwell Lagrangian. In contrast to the quasilinear model studied by O. Kavian and in a previous work in collaboration with G. Figueiredo, this model still gives rise to a linear operator for the electric field whose fundamental solution is known. Nevertheless, also in this case truncation arguments have to be used  to obtain solutions by Mountain Pass arguments.
	
	\vspace{24pt}
%====================================================================

	\talktitle[Gelson Conceicao Goncalves dos Santos]{Existence of solutions for a NSE with discontinuous nonlinearity}
	
	\authorinfo[cgelson@ymail.com]{Gelson Conceicao Goncalves dos Santos}{Universidade Federal do Pará}
	
	\noindent\textbf{Resumo}.\index{Gelson C. G. Santos}\label{gcgs} 
	Apresentaremos um estudo de existência de solução positiva para uma classe de equação não linear de Schrödinger com não linearidade descontinua e duas novas classes de potencias introduzida recentemente por Alves. Nossas principais ferramente para este estudo foram o método de penalização de Del Pino e Felmer e a teoria dos pontos críticos para funcionais localmente Lipschitz. 
	
	\vspace{24pt}
	%====================================================================
	\talktitle[Gustavo Madeira]{Concave-convex structure for a nonlocal and nonhomogeneous problem}
	
	\authorinfo[gfmadeira@dm.ufscar.br]{Gustavo Madeira}{Universidade Federal de São Carlos}
	
	\noindent\textbf{Resumo}.\index{Gustavo Madeira}\label{gm} 
	We are concerned in this lecture with existence of multiple solutions for a class of nonlocal and nonhomogeneous elliptic problems. The nonlocal term is a Kirchhoff type term and the operator includes several examples like p-laplacian, $p\&q$-laplacian, p-mean curvature, among others appearing in the applications. A particular important example of source term is a combination of convex and concave functions. The results to be discussed establish the existence of infinitely many solutions of negative energy (which converge to zero uniformly) and infinitely many solutions of positive energy. The existence of at least two positive solutions to the problem will be also proved. In the final part of the lecture some extentions will be also discussed.
	
	\vspace{24pt}
%====================================================================
	\talktitle[Henrique Renn\'o Zanata]{On a Kirchhoff-Schr\"odinger equation in $\mathbb{R}^2$ involving critical exponential growth}
	
	\authorinfo[henriquerz@hotmail.com]{Henrique Renn\'o Zanata}{Universidade de Bras\'ilia}
	
	\noindent\textbf{Abstract.}\index{Henrique R. Zanata}\label{hz}
	We will present existence results of nonnegative ground state solution to the problem
	$$
	m\left(\int_{\mathbb{R}^2}(|\nabla u|^2 + b(x)u^2) \textrm{d}x\right)(-\Delta u + b(x)u) = A(x)f(u) \ \ \ \textrm{in} \ \ \ \mathbb{R}^2,
	\leqno{(P)}
	$$
	where $m : [0,\infty) \rightarrow (0,\infty)$ and $f : \mathbb{R} \rightarrow [0,\infty)$ are continuous functions and $b,A \in L_{\textrm{loc}}^{\infty}(\mathbb{R}^2)$. The potential $b$ can be negative or vanish on sets with positive measure and the nonlinearity $f$ has critical growth in the sense of Trudinger-Moser inequality. We consider suitable assumptions on $b$, $A$ and $f$ that allow us to treat this problem variationally in the space 
	$$
	H := \left\{u \in W^{1,2}(\mathbb{R}^2) : \int_{\mathbb{R}^2} b(x)u^2 \ \textrm{d}x < \infty\right\}.
	$$
	The solution is obtained from the Mountain Pass Theorem.  
	
	\vspace*{0.5cm} \noindent Joint work with Marcelo F. Furtado (UnB).
	
	\vspace{24pt}
%====================================================================

	\talktitle[Jefferson Abrantes dos Santos]{Generalized $N$-Laplacian equations involving critical exponential growth and concave terms in $\mathbb{R}^N$}
	
	\authorinfo[jeffer.abrantes@gmail.com ]{Jefferson Abrantes dos Santos}{Universidade Federal de Campina Grande}
	
	\noindent\textbf{Resumo}.\index{Jefferson A. Santos}\label{jas} 
	In this work we establish the existence and multiplicity of nonzero and nonnegative solutions for a class of quasilinear elliptic equations, known as Generalized $N$-Laplacian, whose nonlinearity is allowed to enjoy the critical exponential growth with respect to a version of the Trudinger-Moser inequality and it can also contain concave terms in $\mathbb{R}^N$ $(N\geq 2)$. In order to obtain our results, we combine variational arguments in a suitable subspace of a Orlicz-Sobolev space with a version of the Trudinger-Moser inequality and Ekeland Variational Principle. In a particular case, we show that the solution is a positive ground state.

	\vspace*{0.5cm} \noindent Joint work with Uberlandio B. Severo (DM/UFPB).
	
	\vspace{24pt}
%====================================================================

	\talktitle[Juliana Pimentel]{Unbounded attractors under perturbations}
	
	\authorinfo[juliana.pimentel@ufabc.edu.br]{Juliana Pimentel}{Universidade Federal do ABC}
	
	\noindent\textbf{Resumo}.\index{Juliana Pimentel}\label{jp} 
	We put forward the recently introduced notion of unbounded attractors. These objects will be addressed in the context of a class of 1-D semilinear parabolic equations. The nonlinearities are assumed to be non-dissipative and, in addition, defined in such a way that the equation possesses unbounded solutions as time goes to infinity. Small autonomous and non-autonomous perturbations of these equations will be treated. 
	
	\vspace*{0.5cm} \noindent This is based on joint work with A. Carvalho and S. Bruschi.
	
	\vspace{24pt}
%====================================================================
	\talktitle[Lais Moreira]{Existence and multiplicity of positive solutions for a class of singular and nonlocal quasilinear problems}

	\authorinfo[matmslais@gmail.com]{Lais Moreira}{Universidade de Bras\'{i}lia}

	\noindent\textbf{Resumo}.\index{Lais Moreira}\label{lm} 
	In this talk, I am going to present branches of  multiplicity and non-existence of positive $W^{1,p}_{loc}(\Omega)$-solutions for the strong singular $\lambda$-problem 
	$$
	-{\Big(\int_\Omega g(u)dx\Big)^r}\Delta_pu={\lambda } \Big(a(x)u^{-\delta} + b(x)u^{\beta}\Big)   \ \ \mbox{in} \ \ \Omega, \ \ \ \ u > 0 \ \ \ \mbox{in} \ \Omega \ \ \ \mbox{and} \ \ u=0 \ \ \mbox{on} \ \partial \Omega$$
	and  $W^{1,p}_{0}(\Omega)$-solutions for the strong singular Kirchhoff $\lambda$-problem
	$$
	-M(\parallel \nabla u\parallel_p^p)\Delta_pu={\lambda } \Big(a(x)u^{-\delta} + b(x)u^{\beta}\Big)   \ \ \mbox{in} \ \ \Omega, \ \ \ \ u > 0 \ \ \ \mbox{in} \ \Omega \ \ \ \mbox{and} \ \ u=0 \ \ \mbox{on} \ \partial \Omega, $$
	where $\|\cdot\|_p$ denotes the norm of $L^{p}(\Omega)$, $\delta >1$, $0 <  \beta < p-1$, $r \in \mathbb{R}$, $0\leq a,b$  are measurable  functions, $0<g$ is a continuous function and $\Omega \subset \mathbb{R}^N $  is a smooth bounded domain. 
	
	Besides this, we present some results to  $\lambda$-problem 
	$$
	-{\Big(\int_\Omega g(u)dx\Big)^r}\Delta_pu={\lambda } f(u)  \ \ \mbox{in} \ \ \Omega, \ \ \ \ u > 0 \ \ \ \mbox{in} \ \Omega \ \ \ \mbox{and} \ \ u=0 \ \ \mbox{on} \ \partial \Omega $$ when no requirement about monotonicity under $f$ is done.
	
	Our approach is based on  sub-supersolutions techniques,  bifurcation theory and a Comparison Principle proved by us for sub-supersolutions in  $W^{1,p}_{loc}(\Omega)$ for a $p$-Laplacian problem perturbed by a singular term at zero and by a $(p-1)$-sublinear term at infinity.

	\vspace*{0.5cm} \noindent This is a joint work with Carlos Santos (UnB).
	
\selectlanguage{english}

\begin{thebibliography}{999}
	\bibitem{ALVES} Alves, C. O, Covei, D., Existence of solution for a class of nonlocal elliptic problem via
	sub-supersolution method, Nonlinear Analysis: Real World Applications 23 (2015) 1-8.
	
	\bibitem{ARCOYA} Arcoya, D., Leonori, T., Primo, A., Existence of solutions for semilinear nonlocal elliptic
	problems via a Bolzano theorem, Acta Appl Math 127 (2013), 87-104.
	
	\bibitem{CRA}  Crandall, M. G., Rabinowitz,  P. H.,  Tartar, L.,  On a Dirichlet problem with a singular nonlinearity, Comm. Partial Differential Equations  2 (1977), 193-222.
	\bibitem{DU} Du, Y., Bifurcation from infinity in a class a nonlocal elliptic problems, Differential and
	Integral Equations 15 n 5 (2002), 587-606.
	
	\bibitem{MELIAN} Garc\'ia-Meli\'an, J., Lis, J. S., A boundary blow-up problem with a nonlocal reaction, Nonlinear Analysis 75 (2012), 2774-2792.
\end{thebibliography}

\vspace{24pt}
%====================================================================
	\talktitle[Leandro Tavares]{A sub-supersolution method for a class of nonlocal problems involving the p(x)-Laplacian operator and applications}
	
	\authorinfo[leoibilce@hotmail.com]{Leandro Tavares}{Universidade Federal do Cariri}
	
	\noindent\textbf{Resumo}.\index{Leandro Tavares}\label{lt} 
	In this work we are interested in the nonlocal problem 
	$$
\left\{\begin{array}{rcl}\label{problema-(P)}
-\mathcal{A}(x,|u|_{L^{r(x)}})\Delta_{p(x)} u&=&f(x,u)|u|_{L^{q(x)}}^{\alpha(x)}+g(x,u)|u|_{L^{s(x)}}^{\gamma(x)} \;\;\mbox{in} \;\;\Omega,\\
\vspace{.2cm}
u&=&0\;\;\mbox{on}\;\;\partial\Omega,
\end{array}
\right. \eqno{(P)}
$$
	where $\Omega$ is a bounded domain in ${\mathbb{R}}^N (N >1)$ with  $C^{2}$ boundary, $|.|_{L^m(x)}$ is the norm of the space $L^{m(x)}(\Omega),$ $-\Delta_{p(x)}u:=-div(|\nabla u|^{p(x)-2}\nabla u)$ is the $p(x)-$Laplacian operator, $r,q, s,\alpha,\gamma:\Omega\rightarrow[0,\infty)$ are measurable functions and $\mathcal{A},f,g:\overline{\Omega}\times\mathbb{R}\rightarrow\mathbb{R}$ are continuous functions satisfying certain conditions.

	\vspace*{0.5cm} \noindent This work was done in collaboration with Gelson C. G. dos Santos (UFPA) and Giovany M. Figueiredo (UnB)
	
	\vspace{24pt}
%====================================================================
	\talktitle[Marcelo Carvaho Ferreira]{Multi-bump solutions for a class of quasilinear problems involving variable exponents}
	
	\authorinfo[marcelo@mat.ufcg.edu.br]{Marcelo Carvalho Ferreira}{Universidade Federal de Campina Grande}
	
	\noindent\textbf{Resumo}.\index{Marcelo C. Ferreira}\label{mcf} 
	We establish the existence of multi-bump solutions for the following class of quasilinear problems	
	$$ -\Delta_{p(x)}u + (\lambda V(x)+Z(x))u^{p(x)-1}=f(x,u), \, u\geq 0 \ \text{in}\ \mathbb R^N,$$
	where the nonlinearity $f\colon\mathbb R^N\times\mathbb R\to\mathbb R$ is a continuous function having a subcritical growth and the potentials $V,Z\colon \mathbb R^N\to\mathbb R$ are continuos functions verifying some hypotheses. The main tool used is the variational method.
	
	\vspace*{0.5cm} \noindent This work was done in collaboration with Claudianor O. Alves (UFCG)
	
	\vspace{24pt}
	%====================================================================
	
	\talktitle[Marcia Cristina A. B. Federson]{New trends of the theory of nonabsolute integration}
	
	\authorinfo[federson@icmc.usp.br]{Marcia Cristina A. B. Federson}{Universidade de São Paulo (ICMC - USP)}
	
	\noindent\textbf{Resumo}.\index{Marcia C. A. B. Federson}\label{mcabf} 
	We present new results and applications of the theory of nonabsolute integration.
	
	\vspace{24pt}
	%====================================================================
	
	\talktitle[Marcos Leandro Mendes Carvalho]{A Br\'ezis-Oswald problem to $\Phi-$Laplacian operator with a gradient and singular term}
	
	\authorinfo[marcos\_leandro\_carvalho@ufg.br]{Marcos Leandro Mendes Carvalho}{Universidade Federal de Goiás}
	
	\noindent\textbf{Resumo}.\index{Marcos L. M. Carvalho}\label{mlmc} 
	It is establish existence of solution and minimal solution in $W^{1,\Phi}_{loc}(\Omega)$ to the quasilinear elliptic problem
	$$
	\left\{
	\begin{array}{l}
	\displaystyle-\Delta_\Phi u= \lambda f(x,u)+\mu\vert \nabla u \vert^{\sigma}\mbox{in}\Omega,\\
	u>0~\mbox{in}\Omega,~u=0\mbox{on}~\partial \Omega,
	\end{array}
	\right.
	$$
	where $f$ has a sublinear growth, $\sigma>0$ is an appropriate power, $\lambda>0$, and $\mu \geq 0$ are real parameters. Our results are an improvement of the classical Br\'ezis-Oswald result to Orlicz-Sobolev setting by including singular nonlinearity as well as a gradient term.
	
	
	
	\vspace*{0.5cm} \noindent This is a joint word with: J. V. Goncalves, E. D. da Silva and C. A. Santos.
	
	\selectlanguage{english}
	
	\begin{thebibliography}{999}
	\bibitem{Carvalho} Carvalho, M.L., Goncalves, J.V., da Silva, E.D., C. A. {\it A Br\'ezis-Oswald problem to $\Phi\!-\!Laplacian$ operator in the presence of singular terms}, preprint.
	
	\bibitem{Djairo} de Figueiredo, D. G., Gossez,  J-P., Quoirin, H. R., Ubilla, P., {\it Elliptic equations involving the p-Laplacian and a gradient term having natural growth}  Calc. Var. Partial Differential Equations 56(2), 56--32, (2017). 
	\end{thebibliography}
	
	\vspace{24pt}
	%====================================================================
	
	\talktitle[Marcos Tadeu Oliveira Pimenta]{On Existence and concentration of solutions to a class of quasilinear problems involving the 1-Laplacian operator}
	
	\authorinfo[mtopimenta@gmail.com]{Marcos Tadeu Oliveira Pimenta}{Universidade Estadual Paulista}
	
	\noindent\textbf{Resumo}.\index{Marcos T. O. Pimenta}\label{mtop} 
	In this work we use variational methods to prove results on existence and  concentration of solutions to a problem in $\mathbb{R}^N$ involving the 1-Laplacian operator. A thorough analysis on the energy functional defined in the space of functions of bounded variation is necessary, where the lack of compactness is overcome by using the Concentration of Compactness Principle of Lions. 
	
	\vspace{24pt}

%====================================================================
	\talktitle[Ol\'{\i}mpio  Hiroshi Miyagaki]{A class of non resonant fractional critical elliptic problem}
	
	\authorinfo[ohmiyagaki@gmail.com]{Ol\'{\i}mpio  Hiroshi Miyagaki}{Universidade Federal de Juiz de Fora}
	
	\noindent\textbf{Resumo}.\index{Ol\'{\i}mpio  H. Miyagaki}\label{ohm} 
	In this talk we will discuss: the existence, non existence and multiplicity of solutions for
	$$(-\Delta)^s u= a u + g(x,u) \ \mbox{in} \ \Omega \subset \mathbb{R}^N,$$
	where $ a>0$, $ a \neq \lambda_{js}$ and $ g$ is $C^1-$ function, and
	$\lambda_{js}$ is the $j-$th eigenvalue of $(-\Delta)^s$
	The results extend or complement, mainly, that results in the papers \cite{ref5}, \cite{ref6} and \cite{ref7}.
	
	\selectlanguage{english}
	
	\begin{thebibliography}{999}
		\bibitem{ref5} de Paiva, Francisco Odair; Presoto, Adilson E. Semilinear elliptic problems with asymmetric nonlinearities. J. Math. Anal. Appl. 409 (2014), no. 1, 254–262.
		
		\bibitem{ref6}M. Calanchi, B. Ruf.Elliptic equations with one-sided critical growth Electron. J. Differential Equations, 89 (2002), p. 21.
		
		\bibitem{ref7}F. Gazzola, B. Ruf, Lower-order perturbations of critical growth nonlinearities in semilinear elliptic equations, Adv. Differential Equations, 2 (1997), 555-572.
	\end{thebibliography}
	
	\vspace{24pt}
%====================================================================
	
	\talktitle[Paolo Piccione]{Existence of multiple solutions for the van der Waals-Allen-Cahn equation}
	
	\authorinfo[piccione.p@gmail.com]{Paolo Piccione}{Universidade de São Paulo (IME-USP)}
	
	\noindent\textbf{Resumo}.\index{Paolo Piccione}\label{pp} 
	I will discuss the existence of multiple solutions for the following nonlinear problem: for a fixed $V\in \mathds{R}^{+}=\left]0,+\infty\right[$ and $\varepsilon>0$ small, find $
	u\in H_{0}^{1}(\Omega )$, and $\lambda \in \mathds{R}$ such that$\ $ 
	\[
	-\varepsilon ^{2}\Delta u+W^{\prime }(u) =\lambda,
	\]
	and
	\[
	\int_{\Omega }u(x)\,\mathrm dx =V,\]
	where $\Omega$ is an open bounded set in $\mathds{R}^{N}$ and $W:\mathds R\to\mathds R$ is a function of class  $C^2$  which satisfies the following assumptions:
	\begin{itemize}
		\item[(a)] $W(0)=W'(0)=0$,  $W''(0)>0$;
		\item[(b)] there exists $s_0\in\left]0,+\infty\right[$ such that 
		\[W(s_0)=\min\big\{W(s):s\in\mathds{R}\big\}<0;\]
		\item[(c)] suitable growth conditions.
	\end{itemize}
	The simplest example of this type of potentials is given by the non-symmetric Allen-Cahn potential: 
	\[
	W(s)=s^2(s-s_1)(s-s_2),
	\]
	where $0<s_1<s_0<s_2$. In theoretical biology equations of this type model pattern formation related
	to solutions which are not absolute minima of the energy. From a purely mathematical viewpoint, the above equation is also interesting due to its relation with the theory of constant mean curvature hypersurfaces.
	
	\vspace*{0.5cm} \noindent This is a joint work with V. Benci and S. Nardulli. 
	
	\vspace{24pt}
%====================================================================
	\talktitle[Paulo Mendes de Carvalho Neto]{Navier-Stokes equations with time fractional derivative}
	
	\authorinfo[paulo.carvalho@ufsc.br]{Paulo Mendes de Carvalho Neto}{Universidade Federal de São Carlos}
	
	\noindent\textbf{Resumo}.\index{Paulo M. C. Neto}\label{pmcn} 
	The aim of this lecture is to analyse the generalized NavierStokes equations with time fractional dierential operator:
	\begin{eqnarray*}
	cD^\alpha_t u-v\Delta u+(u \cdot \nabla)u+\nabla p &=& f \quad \mbox{ in } \mathbb{R}^N , t >0,\\
	\nabla\cdot u &=& 0  \quad \mbox{ in } \mathbb{R}^N , t>0, \qquad (FNS)\\
	u(x,0) &=& u_0 \quad \mbox{ in } \mathbb{R}^N ,
	\end{eqnarray*}
	where $\alpha \in (0, 1)$ is a fixed number and $cD^\alpha_t$ is the Caputo fractional derivative. More specially, we address this matter using the theory of fractional abstract Cauchy problems, proving that it possesses an unique global mild solution with certain interesting decay properties. Then we discuss the integrability in time of this solution and show that it has a non expected regularity. Finally, we use all the obtained information to relate these fractional solutions with the classical one.

	\vspace*{0.5cm} \noindent This is a joint work with Gabriela Planas.
	
	\vspace{24pt}
%====================================================================
	\talktitle[Pierluigi Benevieri]{A global bifurcation theorem for critical values of $C^1$ maps in Banach spaces}
	
	\authorinfo[pluigi@ime.usp.br]{Pierluigi Benevieri}{Universidade de São Paulo (IME- USP)}
	
	\noindent\textbf{Resumo}.\index{Pierluigi Benevieri}\label{pb} 
	We prove the existence of a global bifurcation branch of critical values of a $C^1$ map $f : \mathbb{R} \times X \to \mathbb{R}$, where $X$ is a real Banach space, and some topological conditions are verified. This result includes the particular case when $X$ is a separable Hilbert space and the spectral flows of suitable paths of the Hessian operators of $f$ are well defined and nonzero. 

	\vspace*{0.5cm} \noindent This is a joint work with P. Amster e J. Haddad.
	
	\vspace{24pt}
%====================================================================
	\talktitle[Pietro D'Avenia]{Some aspects of the Born-Infeld equation}
	
	\authorinfo[pietro.davenia@poliba.it]{Pietro d'Avenia}{Politecnico di Bari}
	
	\noindent\textbf{Resumo}.\index{Pietro d'Avenia}\label{pa} 
	
	We consider the equation
	\begin{equation}\label{eq:BIabs}
		\tag{$\mathcal{BI}$}
		\begin{cases}
			-\operatorname{div}\left(\displaystyle\frac{\nabla
				\phi}{\sqrt{1-|\nabla \phi|^2}}\right)= \rho, \& x\in\mathbb{R}^N,
			\\
			\displaystyle\lim_{|x|\to \infty}\phi(x)= 0.
		\end{cases}
	\end{equation}
	that appears in the purely electrostatic case of the Born-Infeld nonlinear electromagnetic theory.
	In particular $\phi$ is the electric potential, $\rho$ is an assigned extended charge density, and \eqref{eq:BIabs} corresponds to the Gauss law (or Poisson equation) in the classical Maxwell theory.\\  
	We discuss existence of solutions $\phi_\rho$, existence of {\em equilibrium distributions} $\hat{\rho}$, namely distributions that create least-energy potentials among all possible charge distributions, and properties of the corresponding {\em equilibrium potentials} $\phi_{\hat{\rho}}$ for \eqref{eq:BIabs}.\\
	
	\vspace*{0.5cm} \noindent The results have been obtained in joint works with Denis Bonheure, Alessio Pomponio, and Wolfgang Reichel.
	
	
	\vspace{24pt}
%====================================================================
	\talktitle[Raquel Lehrer]{Blow up type and existence results of solutions for an asymptotically linear nonlocal hyperbolic equation}
	
	\authorinfo[rlehrer@gmail.com]{Raquel Lehrer}{Universidade Estadual do Oeste do Paraná}
	
	\noindent\textbf{Resumo}.\index{Raquel Lehrer}\label{rl} 
	In this talk we present a study of  the behaviour of solutions for an asymptotically linear nonlocal hyperbolic equation. We used the Pohozaev manifold combined with a new technique to find a subspace of $H^s$ ($R^n$ ) where the solution blow up in an appropriate sense. We also proved the global existence of solution in another subspace of $H^s$ ($R^n$ ).
	
	\vspace{24pt}

%====================================================================

	\talktitle[R\'ubia G. Nascimento]{Existence of positive solutions for a class of $p\& q$ elliptic problem with critical exponent and discontinuous nonlinearity}
	
	\authorinfo[rubia\_ufpa@yahoo.com.br]{R\'ubia G. Nascimento}{Universidade Federal do Pará}
	
	\noindent\textbf{Resumo}.\index{R\'ubia G. Nascimento}\label{rbn} 
	In this paper we study the existence of positive solutions to a class of  $p \& q$ elliptic problems given by
$$
-\mbox{div}(a(|\nabla u|^{p})|\nabla u|^{p-2}\nabla u)= f(u)+|u|^{q^{*}-2}u \ \mbox{in}\Omega,\ \  u=0\ \mbox{on} \ \ \partial\Omega,
$$
where  $\Omega\subset\mathbb{R}^{N}$ is bounded, $2 \leq p \leq q< q^{*}$, $f:\mathbb{R}\rightarrow \mathbb{R}$ is a function that can have an uncountable set of discontinuity points and the function $a$ is a continuous function. This result to extend previous ones to a larger class of $p\&q$ type problems.
	
	\vspace{24pt}

%====================================================================
	\talktitle[Willian Cintra da Silva]{Coexistence States in a Cross-Diffusion System of a Predador-Prey Model with Predator Satiation Term}
	
	\authorinfo[willian\_matematica@hotmail.com]{Willian Cintra da Silva}{Universidade de Bras\'{i}lia}
	
	\noindent\textbf{Resumo}.\index{Willian C. Silva}\label{wcs} 
	We are going to talk about existence and non-existence of coexistence states for a cross-difusion system arising from a prey-predator model with a predator satiation term. We use mainly bifurcation methods and a priori bounds to obtain our results. This leads us to study the coexistence region and compare our results with the classical linear difusion predator-prey model. Our results suggest that when there is no abundance of prey, the predator needs to be a good hunter to survive.
	
	\vspace*{0.5cm} \noindent Work developed with collaboration of C. Morales-Rodriogo and A. Suárez (Universidad de Sevilla - Spain).
	
	\vspace{24pt}

%====================================================================
%	Fim dos Resumos de Analise
%====================================================================	


\clearpage	