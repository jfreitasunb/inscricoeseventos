%====================================================================
%	Resumos de Sistemas Dinamicos
%====================================================================

\pagestyle{fancy}

\lhead{}
\chead{{X Workshop de Ver\~{a}o em  Matem\'{a}tica - MAT/UnB}}
\rhead{}
%P\'{a}gina \arabic{page} de \pageref{ultimapagina}

\lfoot{Sistemas Din\^{a}micos}
\cfoot{\arabic{page}}
\rfoot{Ver\~{a}o \ano}

\renewcommand{\headrulewidth}{0.4pt}
\renewcommand{\footrulewidth}{0.4pt}

%====================================================================
%====================================================================

\begin{center}
	%\vspace{1cm}
	\huge{{\bf Sistemas Din\^{a}micos}}
	\vspace{1cm}
\end{center}

%====================================================================
%	Inicio dos Resumos de Sistemas Dinamicos
%====================================================================
	
	\talktitle[Christian S. Rodrigues]{Optimal Transport in Dynamical Systems}
	
	\authorinfo[rodrigues@ime.unicamp.br]{Christian S. Rodrigues}{Universidade Estadual de Campinas}
	
	\noindent\textbf{Resumo}.\index{Christian S. Rodrigues}\label{csr} 
	I shall introduce ideas from the flourishing area of Optimal Transport Theory in order to tackle problem of ergodic theory. In particular, we shall address the representation of measures and stochastic properties of dynamical systems.
	
	\vspace{24pt}

%====================================================================
	
	\talktitle[Dahisy Valadão de Souza Lima]{Cancellation of Singularities in the Circle-valued Morse Theory}
	
	\authorinfo[dahisylima@gmail.com]{Dahisy Valadão de Souza Lima}{Universidade Estadual de Campinas}
	
	\noindent\textbf{Resumo}.\index{Dahisy V. S. Lima}\label{dvsl} 
	We define a topological context fruitful in obtaining information on
	the behaviour of a wide range of dynamical systems. The overarching idea is to define an appropriate filtered chain complex which captures connections between the invariant sets of the system.
	With these tools, we consider as our major algebraic apparatus a
	spectral sequence associated to the given filtered chain complex. The unfolding of the spectral sequence exhibits a rich algebraic procedure and provides much insight into dynamical properties of
	a continuation of the dynamical systems being studied, such as
	bifurcation phenomena due the cancellation of singularities. Our goal is to present this algebraic-dynamical set-up at the context
	of circle-valued Morse functions $f$ on a closed 2-manifold, where we consider the Novikov chain complex associated to $f$. We use the data of the spectral sequence to keep track of all dynamical information on the birth and death of connecting orbits between consecutive critical points, as well as periodic orbits that may arise within a family of circle-valued Morse functions. Furthermore, we show that this family corresponds to a continuation from the initial Morse-Novikov flow to a minimal Morse-Novikov flow.
	
	\vspace{24pt}	
%====================================================================

	\talktitle[Hugo Tadashi Muniz Kussaba]{Estabilização de pose de objetos rígidos usando quaternions duais unitários e sistemas dinâmicos híbridos}
	
	\authorinfo[kussaba@lara.unb.br]{Hugo Tadashi Muniz Kussaba}{Universidade de Bras\'{i}lia}
	
	\noindent\textbf{Resumo}.\index{Hugo T. M. Kussaba}\label{htmk} 
	Usando a teoria de sistemas dinâmicos híbridos proposta 
	por Andrew R. Teel, Rafal Goebel e Ricardo G. Sanfelice. Em especial, será mostrado alguns 
	resultados de um trabalho recente publicado em conjunto com Luis F. C. Figueredo, João Y. Ishihara
	e Bruno V. Adorno.  
	
	Motivado por aplicações em robótica e sistemas mecânicos de modo geral, vários trabalhos da literatura de teoria de controle consideram problemas de estabilização de 	sistemas dinâmicos definidos em grupos de Lie, em particular os grupos SO(3), SE(3), dos quaternions unitários e dos quaternions duais unitários.  Nesta palestra revisaremos alguns problemas relacionados à estabilização nesses grupos e como mitigar esses problemas usando a teoria de sistemas dinâmicos híbridos proposta por Andrew R. Teel, Rafal Goebel e Ricardo G. Sanfelice. Em especial, será mostrado alguns resultados de um trabalho recente publicado em conjunto com Luis F. C. Figueredo, João Y. Ishihara e Bruno V. Adorno.
	
	\vspace{24pt}
%====================================================================

	\talktitle[Rafaela Prado]{Aspectos variacionais de geodésicas homogêneas em variedades flag}
	
	\authorinfo[rafaelafprado@gmail.com]{Rafaela Prado}{Instituto Federal de Brasília}
	
	\noindent\textbf{Resumo}.\index{Rafaela Prado}\label{rp} 
	Estudaremos pontos conjugados ao longo de geodésicas homogêneas em variedades flag generalizadas a partir da análise da segunda variação da energia de tais geodésicas. Também daremos um exemplo de como o fluxo de Ricci pode evoluir de forma a produzir pontos conjugados no espaço projetivo complexo visto como uma Sp(n+1)-variedade Riemanniana homogênea.
	
	\vspace*{0.5cm} \noindent Este trabalho é conjunto com Lino Grama (UNICAMP).
	\vspace{24pt}
%====================================================================

%	\talktitle[Autor]{Titulo}
%	
%	\authorinfo[email@unb.br]{Autor}{Universidade de....}
%	
%	\noindent\textbf{Resumo}.\index{Autor}\label{Autor} 
%	Dizemos.......... 
%	
%	\vspace{24pt}
%====================================================================

%	\talktitle[Autor]{Titulo}
%	
%	\authorinfo[email@unb.br]{Autor}{Universidade de....}
%	
%	\noindent\textbf{Resumo}.\index{Autor}\label{Autor} 
%	Dizemos.......... 
%	
%	\vspace{24pt}
%====================================================================

%	\talktitle[Autor]{Titulo}
%	
%	\authorinfo[email@unb.br]{Autor}{Universidade de....}
%	
%	\noindent\textbf{Resumo}.\index{Autor}\label{Autor} 
%	Dizemos.......... 
%	
%	\vspace{24pt}
%====================================================================

%	\talktitle[Autor]{Titulo}
%	
%	\authorinfo[email@unb.br]{Autor}{Universidade de....}
%	
%	\noindent\textbf{Resumo}.\index{Autor}\label{Autor} 
%	Dizemos.......... 
%	
%	\vspace{24pt}

%====================================================================
%	Fim dos Resumos de Sistemas Dinamicos
%====================================================================	


\clearpage	