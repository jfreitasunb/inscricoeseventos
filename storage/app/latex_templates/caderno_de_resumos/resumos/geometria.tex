%====================================================================
%	Resumos de Geometria
%====================================================================

\pagestyle{fancy}

\lhead{}
\chead{{X Workshop de Ver\~{a}o em  Matem\'{a}tica - MAT/UnB}}
\rhead{}
%P\'{a}gina \arabic{page} de \pageref{ultimapagina}

\lfoot{Geometria}
\cfoot{\arabic{page}}
\rfoot{Ver\~{a}o \ano}

\renewcommand{\headrulewidth}{0.4pt}
\renewcommand{\footrulewidth}{0.4pt}

%====================================================================
%====================================================================

\begin{center}
	%\vspace{1cm}
	\huge{{\bf Geometria}}
	\vspace{1cm}
\end{center}

%====================================================================
%	Inicio dos Resumos de Geometria
%====================================================================


	\talktitle[Benedito Leandro Neto]{Invariant Solutions for the Static Vacuum Equation}
	
	\authorinfo[bleandroneto@gmail.com]{Benedito Lerandro Neto}{Universidade Federal de Goi\'{a}s}
	
	\noindent\textbf{Resumo}.\index{Benedito L. Neto}\label{bln} 
	We consider the static vacuum Einstein space-time when the spatial factor (or, base) is conformal to a pseudo-Euclidean space, which is invariant under the action of a translation group. We characterize all such solitons. Moreover, we give examples of static vacuum Einstein solutions for the Einstein's field equation. Applications provide an explicit example of a complete static vacuum Einstein space-time.
	
	
	\vspace{24pt}
%====================================================================

	\talktitle[Claudiano Goulart]{Uma relação entre as transformações de B\"{a}cklund e Ribaucour}
	
	\authorinfo[goulart.fsa@gmail.com]{Claudiano Goulart}{Universidade Estadual de Feira de Santana}
	
	\noindent\textbf{Resumo}.\index{Claudiano Goulart}\label{cg} 
	Dada uma superfície imersa em $\mathbb{R}^3$, com curvatura Gaussiana constante negativa, a composição de transformações de B\"{a}cklund gera uma família a 4 parâmetros de superfícies com a mesma curvatura. Tendo em vista que a transformação de Ribaucour de tais superfícies também fornece uma família a 4 parâmetros do mesmo tipo então é natural questionar se os dois métodos são equivalentes. A resposta, em geral, é negativa. Apresentaremos condições necessárias e suficientes para que as supefícies dadas por estes dois métodos sejam congruentes.  Além disto, partindo com a pseudo esfera, vamos obter um exemplo explicito em que a composição de uma transformação de B\"{a}cklund não é uma transformação de Ribaucour. 
	
	\vspace{24pt}

%====================================================================

	\talktitle[Diego Catalano Ferraioli]{Imersões isométricas locais de superfícies descritas por equações de tipo pseudo-esférico}
	
	\authorinfo[diego.catalanoferraioli@gmail.com]{Diego Catalano Ferraioli}{Universidade Federal da Bahia}
	
	\noindent\textbf{Resumo}.\index{Diego C. Ferraioli}\label{dcf} 
	As equações diferenciais de tipo pseudo-esférico são encontradas em vários âmbitos, tanto na física quanto na matemática. Por definição uma equação que descreve superfícies pseudo-esféricas é equivalente às equações de estrutura de uma superfície com curvatura de Gauss K = -1. O interesse para este tipo de equações começou com a observação de Sasaki \cite{7} de que até então todas as equações que tinham sido integradas com o método do espalhamento inverso, como por exemplo a KdV e a sine-Gordon, eram deste tipo especial. 
	Mais em geral, as equações de tipo pseudo-esférico são exemplos de equações com a importante propriedade de possuir uma representação a curvatura nula, i.e., de ser condição de compatibilidade de um sistema linear de equações a derivadas parciais da primeira ordem (problema linear). De fato, todas as propriedades de integrabilidade dessas equações seguem da existência de um correspondente problema linear. 
	O estudo geral das propriedades geométricas das equações diferenciais de tipo pseudo-esférico foi começado em um trabalho pioneiro de Chern e Tenenblat \cite{3}, no qual foi também indicado um método útil para a classificação de algumas dessas equações. 
	Nesta palestra, discutiremos o problema da existência de imersões isométricas locais para as superfícies pseudo-esféricas descritas por equações deste tipo. Este problema foi considerado pela primeira vez em \cite{5}, e ulteriormente investigado em outros trabalhos \cite{1,2,4,6}. Em particular, após de um apanho geral sobre os resultados encontrados até o momento sobre este problema, discutiremos nossos resultados mais recentes sobre o uso destas imersões para obter transformações de Bäcklund entre equações de tipo pseudo-esférico.
	
	
	
	\begin{thebibliography}{999}
		
		\bibitem{1} T. Castro Silva and N. Kamran. Third-order differential equations and local isometric immersions of pseudospherical surfaces. Commun. Contemp. Math. 18, 1650021 (2016)
			
		\bibitem{2} D. Catalano Ferraioli and L. de Oliveira Silva. Local isometric immersions of pseudospherical surfaces described by evolution equations in conservation law form. Journal of Mathematical Analysis and Applications (Print), v. 446, p. 1606-1631 (2016)
				
		\bibitem{3} S. S. Chern and K. Tenenblat. Pseudospherical surfaces and evolution equations. Stud. Appl. Math. 74, 55–83 (1986)
		
		\bibitem{4} N. Kahouadji, N. Kamran, and K. Tenenblat.  Local Isometric Immersions of Pseudo-spherical Surfaces and k-th Order Evolution Equations.  Arxiv:1701.08004v1 [math.DG] 27 Jan 2017
		
		\bibitem{5} N. Kahouadji, N. Kamran, and K. Tenenblat.  Second-order equations and local isometric immersions of pseudo-spherical surfaces. Comm. Anal. Geom., 24, no. 3, pp. 605-643 (2016)
		
		\bibitem{6} N. Kahouadji, N. Kamran, K. Tenenblat. Local Isometric Immersions of Pseudo-Spherical Surfaces and Evolution Equations. In: Guyenne P., Nicholls D., Sulem C. (eds) Hamiltonian Partial Differential Equations and Applications. Fields Institute Communications, vol 75. Springer, New York, NY (2015)
				
		\bibitem{7} R. Sasaki. Soliton equations and pseudospherical surfaces. Nucl. Phys. B 154, 343-357 (1979)
		
	\end{thebibliography}
	
	
	\vspace{24pt}
%====================================================================
	
	\talktitle[Marcelo Souza]{Fundamentos de Geometria: Axioma da Régua infinita de Randers}
	
	\authorinfo[msouza\_2000@yahoo.com]{Marcelo Souza}{Universidade Federal de Goi\'{a}s}
	
	\noindent\textbf{Resumo}.\index{Marcelo Souza}\label{ms} 
	In this work, we study concrete models of  non euclidean Geometries where the Cartesian plane is endowed first with a special Randers metric and then with a Randers metric that solves the Zermelo Navigational Problem, under some wind $W$. Here the lines are geodesic then we use that fact to establish a bijection between the line and the Real set numbers, in a such way that the distance from $A$ to $B$  is given by the difference of the respective real numbers associated to this points. The distance is not reversible, but the navigation along the perimeter of any oriented triangle independent on the wind $W$. 

	\vspace*{0.5cm} \noindent Joint work with Newton Mayer Sol\'{o}rzano Ch\'{a}vez (UNILA).

	\selectlanguage{english}
	\begin{thebibliography}{999}
		\bibitem{CRobles}{Colleen Robles}-\textit{Geodesics in Randers spaces of constant curvature}, Trans AMS {359}  (2007), no.  4, 1633 - 1651.
		
		\bibitem{DBaoCRoblesZShen}{David Bao, Colleen Robles and Zhongmin Shen}-\textit{Zermelo navigation on Riemannian manifolds}, J. Diff. Geom. { 66} (2004), 391-449.
		
		\bibitem{PFinsler}{Paul Finsler}- \textit{$\ddot U$ber Kurven und Fl$\ddot a$chen in allgemeinen R$\ddot a$umen}, Dissertation, Gottingen, JFM {46.1131.02} (1918) (Reprinted by Birkhauser (1951))
		
		\bibitem{GRanders}{Randers G.} -\textit{On an asymmetric metric in the four-space of general relativity.} Phys. Rev. {59} (1941) 195-199.
		\bibitem{Zermelo}{Zermelo E.}- \textit{$\ddot U$ber das Navigationsproblem bei ruhender oder veranderlicher Windverteilung}. Zeitschrift fur Angewandte Mathematik und Mechanik. {11} (1931), $114 - 124$.
		
		\bibitem{MSouzaKTenenblat}{M. A. Souza and K. Tenenblat} - \textit{Minimal surfaces of rotation in Finsler space with a Randers metric}, Math. Ann. {325} (2003), no. 4, 625-642.
	\end{thebibliography}
	
	\vspace{24pt}
%====================================================================

	\talktitle[Newton Mayer Sol\'{o}rzano Ch\'{a}vez]{On flag curvature of spherically symmetric Finsler metrics}
	
	\authorinfo[newton.chavez@unila.edu.br]{Newton Mayer Sol\'{o}rzano Ch\'{a}vez}{Universidade Federal da Integração Lantino-Americana}
	
	\noindent\textbf{Resumo}.\index{Newton M. S. Ch\'{a}vez}\label{nmsc} 
	In this work, we classify the spherically symmetric Douglas metrics on a symmetric space $\Omega$ with constant flag curvature and we characterize these metrics with some restrictions.

\selectlanguage{english}
\begin{thebibliography}{999}
	\bibitem{BaoRobles}{D. Bao, C. Robles and Z. Shen}-\textit{Zermelo Navigation on Riemannian manifolds}, J. Diff. Geom. 66(2004), 391-449.
	\bibitem{moszhou} X. Mo and L. Zhou, {\em The curvatures of spherically symmetric finsler metrics in $R^n$}, arXiv: 1202.4543v3 (2012). 
	\bibitem{MoZhouZhu2013} X. Mo, L. Zhou and H. Zhu, {\em on spherically symmetric Finsler metrics of constant curvature}, preprint, 18 Nov. 2013 $R^n$.
\end{thebibliography}
	\vspace{24pt}

%====================================================================

	\talktitle[Romildo da Silva Pina]{Superfícies com curvatura constante em espaços conformemente planos}
	
	\authorinfo[romildo@ufg.br]{Romildo da Silva Pina}{Universidade Federal de Goi\'{a}s}
	
	\noindent\textbf{Resumo}.\index{Romildo S. Pina}\label{rsp} 
	Nesta palestra, vamos introduzir uma família de variedades Riemannianas tridimensionais, que são conformes ao espaço euclidiano, e estudar uma classe de superfícies imersas nesses espaços. 
	
	\vspace{24pt}


%====================================================================
%	Fim dos Resumos de Geometria
%====================================================================	


\clearpage	