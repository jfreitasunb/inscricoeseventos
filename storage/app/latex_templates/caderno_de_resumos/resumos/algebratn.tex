%====================================================================
%	Resumos de Algebra e Teoria dos Numeros
%====================================================================

\pagestyle{fancy}

\lhead{}
\chead{{X Workshop de Ver\~{a}o em  Matem\'{a}tica - MAT/UnB}}
\rhead{}
%P\'{a}gina \arabic{page} de \pageref{ultimapagina}

\lfoot{\'{A}lgebra e Teoria dos N\'{u}meros}
\cfoot{\arabic{page}}
\rfoot{Ver\~{a}o \ano}

\renewcommand{\headrulewidth}{0.4pt}
\renewcommand{\footrulewidth}{0.4pt}

%====================================================================
%====================================================================

\begin{center}
	%\vspace{1cm}
	\huge{{\bf \'{A}lgebra e Teoria dos N\'{u}meros}}
	\vspace{1cm}
\end{center}

%====================================================================
%	Inicio dos Resumos de Algebra e Teoria dos Numeros
%====================================================================

	\talktitle[Alex Carrazedo Dantas]{FC-grupo com poucas órbitas por automorfismos}

	\authorinfo[alexcdan@gmail.com]{Alex Carrazedo Dantas}{Universidade de Bras\'{i}lia}

	\noindent\textbf{Resumo}.\index{Alex C. Dantas} \label{acd} Seja $G$ um grupo. As órbitas da ação de $\mathsf{Aut}(G)$ sobre $G$ são chamadas de órbitas por automorfismos de $G$ e a quantidade de órbitas por automorfismo de $G$ é denotada por $\omega(G)$. Nessa apresentação, vamos mostrar que se $G$ é um FC-grupo com uma quantidade finita de órbitas por automorfismos, então o subgrupo derivado $G'$ é finito e $G$ admite uma decomposição $G = \mathsf{Tor}(G) \times D$, onde $\mathsf{Tor}(G)$ é o subgrupo de torção de $G$ e $D$ é um subgrupo característico e divisível de $\mathsf{Z}(G)$. Também iremos mostrar que se $G$ é um FC-grupo infinito com $\omega(G) \leqslant 8$, então $G$ é solúvel ou $G \cong A_5 \times \mathsf{Z}(G)$. Além disso, iremos descrever a estrutura dos FC-grupos infinitos e não solúveis com no máximo $11$ órbitas por automorfismos. 
%{\noindent} {\bf Keywords:} Tree automorphisms, state-closed representation, Groups of Lamplighter type.

	\vspace*{0.5cm} \noindent Trabalho em conjunto com  \emph{Raimundo A. Bastos}.

\selectlanguage{brazil} 
%	
	\vspace{24pt}

%====================================================================

\talktitle[Alveri Alves Sant'Ana]{Extensões de Hopf-Ore}

\authorinfo[alveri@mat.ufrgs.br]{Alveri Alves Sant'Ana}{Universidade Federal do Rio Grande do Sul}

\noindent\textbf{Resumo}.\index{Alveri A. Sant'Ana}\label{aas} 
Extensões de Ore tem aparecido com frequência no contexto de álgebras de Hopf. Recentemente, Panov introduziu na literatura as extensões Hopf-Ore, caracterizando aquelas que são geradas por elementos skew-primitivos. Se R é uma algebra de Hopf e S é uma extensão de Ore de R, dizemos que S é uma extensão Hopf-Ore de R, se R é uma subálgebra de Hopf de S. Pretendemos nesta palestra apresentar o resultado de Panov, bem como uma generalização do mesmo para o contexto de álgebras de Hopf fracas.

\vspace*{0.5cm} \noindent Obtido em parceria com Christian Lomp e Ricardo Santos.

\vspace{24pt}


%====================================================================

\talktitle[Ana Cristina Vieira]{Algebras with involution and bounded colength}

\authorinfo[acvufmg2011@gmail.com]{Ana Cristina Vieira}{Universidade Federal de Minas Gerais}

\noindent\textbf{Resumo}.\index{Ana C. Vieira}\label{acv} 
In the last years several authors have been studying the behavior of a special character associated to algebras endowed with involution, called $*$-cocharacter. In this talk we exhibit the decomposition of the $*$-cocharacter for some important algebras with involution and compute the number of irreducibles appearing in that decomposition to form the sequence of $*$-colengths. As a consequence we classify the algebras with involution such that the sequence of $*$-colengths is bounded by three.

\vspace*{0.5cm} \noindent This is a joint work with D. La Mattina and T. Nascimento.

\vspace{24pt}
%====================================================================

\talktitle[Andrzej Zuk]{From PDEs to automata groups}

\authorinfo[andrzej.zuk@imj-prg.fr]{Andrzej Zuk}{Université Paris 7}

\noindent\textbf{Resumo}.\index{Andrzej Zuk}\label{az} 
We present a construction with associates to the KdV equation a group generated by an automata. It is related to L2 Betti numbers which are homotopy invariants of closed manifolds.


\vspace{24pt}

%====================================================================
\talktitle[Antonio Marcos Duarte de França]{Sobre An\'{e}is Graduados cuja Componente Neutra é Nil}

\authorinfo[mardua13@gmail.com]{Antonio Marcos Duarte de França}{Universidade de Bras\'{i}lia}

\noindent\textbf{Resumo}.\index{Antonio M. D. França}\label{amdf} 
Sejam $\mathfrak{A}$ uma \'{a}lgebra associativa sobre um corpo $\mathbb{F}$ graduada por um grupo $\mathsf{G}$ e $e$ o elemento neutro de $\mathsf{G}$. \'{E} bem conhecido que se $\mathsf{G}$ é finito e $\mathfrak{A}_e$ \'{e} uma $PI$-\'{a}lgebra, ent\~{a}o $\mathfrak{A}$ \'{e} tamb\'{e}m uma $PI$-\'{a}lgebra. N\'{o}s estudamos um caso espec\'{i}fico deste resultado e respondemos a seguinte quest\~{a}o: o que podemos dizer sobre $\mathfrak{A}$ quando $\mathfrak{A}_e$ \'{e} nil/nilpotente, onde $\mathfrak{A}$ \'{e} um anel associativo (ou uma $\mathbb{F}$-\'{a}lgebra) com uma $\mathsf{G}$-gradua\c{c}\~{a}o? Neste sentido, n\'{o}s estudamos a classe de an\'{e}is (associativos) $\mathsf{G}$-graduados cuja componente neutra \'{e} nil. Sob certas condi\c{c}\~{o}es, n\'{o}s provamos todo anel $\mathsf{G}$-graduado com componente neutra nil \'{e} um anel nil. Entre outros resultados, aplicando o Teorema de Nagata-Higman, n\'{o}s apresentamos uma importante aplica\c{c}\~{a}o de nossos resultados. Al\'{e}m disso, n\'{o}s exibimos uma consider\'{a}vel rela\c{c}\~{a}o entre an\'{e}is graduados e o Problema de K\"{o}the.

\vspace*{0.5cm} \noindent Trabalho em conjunto com Irina Sviridova (UnB).

\vspace{24pt}

%====================================================================
\talktitle[Bruno de Paula Miranda]{Formas diagonais sobre a extensão quadrática não ramificada de $\mathbb{Q}_2$}

\authorinfo[brunodpmiranda@gmail.com]{Bruno de Paula Miranda}{Universidade de Bras\'{i}lia}

\noindent\textbf{Resumo}.\index{Bruno P. Miranda}\label{bpm} 
Em 1963, Davenport e Lewis provaram que se a forma aditiva $f(x) = a_1x_1^d + \cdots + a_Nx_N^d$ com coeficientes em $\mathbb{Q}_p$, o corpo dos números $p$-ádicos, satisfizer $N > d^2$, então existe solução não trivial para $f(x) = 0$. Muito estudo tem sido realizado a fim de generalizar esse resultado para extensões finitas de $\mathbb{Q}_p$. Aqui, estudamos o caso $f(x) \in K[x]$ com $K$ sendo a extensão quadrática não ramificada de $\mathbb{Q}_2$ e verificamos que se $d$ não é potência de 2, então $N> d^2$ garante a existência de solução não trivial para $f(x) = 0$.

\vspace{24pt}
%====================================================================
	
\talktitle[Dessislava H. Kochloukova]{Grupos auto-similares}

\authorinfo[desi@ime.unicamp.br]{Dessislava H. Kochloukova}{Universidade Estadual de Campinas}

\noindent\textbf{Resumo}.\index{Dessislava H. Kochloukova}\label{dehko} Vamos discutir algumas classes de grupos auto-similares que agem transitivamente no primeiro nível da árvore regular e que são de tipo homologico FPm incluindo grupos S-aritméticos e metabelianos.

\vspace*{0.5cm} \noindent Os resultados desta pesquisa foram obtidos junto com Said Sidki (UnB).

	\vspace{24pt}
%====================================================================

	\talktitle[Dimas José Gonçalves]{Central polynomials with involution for $2 \times 2$ upper triangular matrices algebra.}
	
	\authorinfo[dimas@dm.ufscar.br]{Dimas José Gonçalves}{Universidade Federal de São Carlos}
	
	\noindent\textbf{Resumo}.\index{Dimas J. Gonçalves}\label{djg} 
	Let $UT_2(F)$ be the $2 \times 2$ upper triangular matrices algebra over a field $F$ of characteristic different from $2$. Consider an involution of the first kind on $UT_2(F)$. In this talk we will describe the set of all $*$-central polynomials for this algebra.

\vspace*{0.5cm} \noindent This is a joint work with Ronald I. Q. Urure.

	\vspace{24pt}

%====================================================================
	\talktitle[Diogo Diniz Pereira da Silva e Silva]{Identities and Isomorphisms of Upper Block Triangular Matrix Algebras}

	\authorinfo[diogo@mat.ufcg.edu.br]{Diogo Diniz Pereira da Silva e Silva}{Universidade Federal de Campina Grande}

	\noindent\textbf{Resumo}.\index{Diogo D. P. S. Silva}\label{ddpss} 
	Let $(d_1,\dots, d_n)$ be an $n$-tuple of positive integers and $F$ a field. The corresponding algebra $UT(d_1,\dots,d_n)$ of upper block triangular matrices is the subalgebra of $M_{m}(F)$, where $m=d_1+\cdots+d_n$, consisting of the matrices \[
\left(\begin{array}{cccc}
A_{11} & A_{12} & \cdots & A_{1n}\\
0 & A_{22}&\cdots & A_{2n}\\
\vdots & \vdots & \ddots & \vdots \\
0&0&\cdots& A_{nn}
\end{array}\right),
\] 
	where $A_{ij}$ is a block of size $d_i\times d_j$. These algebras play an important role in the classification of minimal varieties of a given exponent (\cite{GZ}). 

	The gradings, by a finite abelian group, on upper block triangular matrix algebras algebras (over an algebraically closed field of characteristic zero) were classified in \cite{VZ}. This classification is in terms of gradings where the elementary matrices are homogeneous, called elementary gradings, and fine gradings on matrix algebras. In this talk we describe the (graded) isomorphism classes of these algebras. Moreover, we describe the graded identities for elementary gradings with commutative neutral component. These are the main results of  \cite{BFD} and \cite{DM}.

	\selectlanguage{english}

\begin{thebibliography}{999}
	\bibitem{BFD} A. R. Borges, C. Fidelis, D. Diniz, \textit{Graded isomorphisms on upper block triangular matrix algebras}, Linear Algebra and its Applications, In Press.
	\bibitem{DM} D. Diniz, T. C. de Mello, \textit{Graded identities of block-triangular matrices}, Journal of Algebra \textbf{464} (2016) 246--265.
	\bibitem{GZ} A. Giambruno, M. Zaicev, \textit{Minimal varieties of algebras of
		exponential growth}, Advances in Mathematics \textbf{174} (2003) 310--323.
	\bibitem{VZ} A. Valenti, M. V. Zaicev, \textit{Abelian gradings on upper block triangular matrices} Canadian Mathematical Bulletin \textbf{55} (2012) 208--213.
\end{thebibliography}

\vspace{24pt}
%====================================================================
	\talktitle[Elena Aladova]{Around Specht problem}

	\authorinfo[aladovael@mail.ru]{Elena Aladova}{Universidade Federal do Rio Grande do Norte}

	\noindent\textbf{Resumo}.\index{Elena Aladova}\label{ea} 
	The present talk is an overview of results concerning to the finite basis problem for associative algebras. This problem was formulated by W.~Specht in 1950 for associative algebras over the field of rational numbers. Afterwards, it was considered for associative and non-associative algebras over an arbitrary fields and at present time it is known as \emph{Specht problem}. We concentrate our attention on joint results with A.N.~Krasilnikov,  in particular, on the finite basis problem for associative algebras satisfying the identity $x^{n}=0$.

	\vspace{24pt}
	%====================================================================
\talktitle[Gláucia Lenita Dierings]{Grupos nos quais as classes de conjuga\c{c}\~{a}o
		contendo comutadores s\~{a}o limitadas}
	
	\authorinfo[glauciadierings@yahoo.com.br]{Gláucia Lenita Dierings}{Universidade de Bras\'{i}lia}
	
	\noindent\textbf{Resumo}.\index{Gláucia L. Dierings}\label{gld} 
	Dado um grupo $G$ e um elemento $x\in G$, escrevemos $x^G$ para a classe de conjugação contendo $x$. Um grupo é dito ser um BFC-grupo se suas classes de conjugação são finitas de tamanho limitado. B. H. Neumann demonstrou que se $G$ é um BFC-grupo, então o grupo derivado $G'$ é finito. O primeiro limitante para a ordem de $G'$ foi encontrado por J. Wiegold em 1957. Nós estamos interessados em grupos nos quais as classes de conjugação contendo comutadores são finitas de tamanho limitado. Obtemos os seguintes resultados:
	
	{\bf Teorema 1}. Se $|x^G|\leq n$ para qualquer comutador $x\in G$, então o segundo grupo derivado $G''$ tem ordem finita $n$-limitada.
	
	{\bf Teorema 2}. Se $|x^{G'}|\leq n$ para qualquer comutador $x\in G$, então $\gamma_3(G')$ tem ordem finita $n$-limitada.
	
	\vspace*{0.5cm} \noindent Este trabalho foi realizado em parceria com Pavel Shumyatsky (UnB).
	
	\vspace{24pt}
	
%====================================================================
	\talktitle[Igor dos Santos Lima]{Virtually free groups and integral representations}

	\authorinfo[igor.matematico@gmail.com]{Igor dos Santos Lima}{Universidade Federal de Goiás}	

	\noindent\textbf{Abstract}.\index{Igor S. Lima}\label{isl} Let $G=F \rtimes H$ be a semidirect product of a free group $F$ and a finite group $H$. The $H$-module structure of the abelianization $F^{ab}$ is described in terms of splitting  of $G$ as the fundamental graph of a graph of finite groups.

	\vspace*{0.5cm} \noindent This is a joint work with Pavel Zalesskii (UnB) accepted for publication in the Journal of Algebra (2017).

\vspace{24pt}
%====================================================================
	
	\talktitle[Ivan Shestakov]{Problema de especialidade para \'{a}lgebras de Malcev}
	
	\authorinfo[ivan.shestakov@gmail.com]{Ivan Shestakov}{Universidade de S\~{a}o Paulo (IME-USP)}
	
	\noindent\textbf{Resumo}.\index{Ivan Shestakov}\label{is} 
	Uma álgebra de Malcev chama-se especial se ela é isomorfa a uma subálgebra de álgebra comutador $A^{(-)}$ para uma álgebra alternativa A. O problema do Malcev, planteada em 1955, se trata de verificar se cada álgebra de Malcev é especial, ou seja, se um análogo do famoso teorema de Poincare-Birkhoff-Witt é valido para as álgebras de Malcev. Nos construímos um exemplo de álgebra de Malcev que não é especial. Assim, o problema do Malcev,  tem uma solução negativa.
	
	\vspace*{0.5cm} \noindent É um trabalho conjunto com A.Buchnev, V.Filippov e S.Sverchkov.
	
	\vspace{24pt}
%====================================================================
	\talktitle[Jean Carlos de Aguiar Lelis]{Sobre alguns problemas relacionados aos números de Liouville}
	
	\authorinfo[jeanlelis.math@gmail.com]{Jean Carlos de Aguiar Lelis}{Universidade de Bras\'{i}lia}
	
	\noindent\textbf{Resumo}.\index{Jean C. A. Lelis}\label{jca} 
	O conjunto dos números de Liouville tem sido objeto de vários estudos devido sua importância para a Teoria dos Números Transcendentes. Os números de Liouville são os primeiros exemplos de números transcendentes, e são números cuja transcendência é mais facilmente demonstrada, e por isso nosso interesse nesse conjunto. Nessa palestra falaremos um pouco sobre alguns problemas relacionados a números de Liouville, e sobre alguns resultados que temos obtido estudando esse conjunto e alguma generalizações dele.
	
	
	\vspace{24pt}
%====================================================================
	
	\talktitle[Jhone Caldeira]{Finite groups admitting automorphisms with nilpotent centralizers}
	
	\authorinfo[jhone@ufg.br]{Jhone Caldeira}{Universidade Federal de Goiás}
	
	\noindent\textbf{Resumo}.\index{Jhone Caldeira}\label{jhone} 
	Let $A$ be a group which acts by automorphisms on a group $G$. We denote by  $C_G(A)$  the centralizer of $A$ in $G$ (the fixed-point subgroup). Very often the structure of $C_G(A)$ has strong influence over the structure of $G$. In particular, some attention was given to the situation where a Frobenius group $FH$ acts by automorphisms on a finite group $G$ (recently prompted by Mazurov's problem 17.72 in the Kourovka Notebook). We present some examples about this phenomenon.
	
	\vspace{24pt}
%====================================================================
	\talktitle[John MacQuarrie]{The path coalgebra as a right adjoint functor}
	
	\authorinfo[john@mat.ufmg.br]{John MacQuarrie}{Universidade Federal de Minas Gerais}
	
	\noindent\textbf{Resumo}.\index{John MacQuarrie}\label{jmq} 
	As Kostiantyn Iusenko (USP) will explain in his short course this week, a finite dimensional algebra $A$ can be well understood via a finite directed graph (known as a quiver) by constructing an algebra from the quiver having $A$ as a well-behaved quotient.  Recently, Kostiantyn and I made the relationship between algebras and quivers functorial, obtaining an informative adjunction.  This relationship passes to inverse limits, by considering "pseudocompact" quivers on one side and pseudocompact algebras on the other. Working with Kostiantyn and Samuel Quirino (USP), a similar adjunction is obtained for abstract quivers and arbitrary coalgebras.  This construction yields a different generalization to the pseudocompact case.  I'll explain all the main ideas involved.
	
	\vspace{24pt}
	%====================================================================
	\talktitle[Juliana Silva Canella]{C\'alculo do grupo $\nu(G)$ de grupos metac\'iclicos}
	
	\authorinfo[jscanella@gmail.com]{Juliana Silva Canella}{Universidade de Bras\'{i}lia}
	
	\noindent\textbf{Resumo}.\index{Juliana S. Canella}\label{jsc} 
	Seja $G$ um grupo metac\'iclico (infinito e finito). Nesta palestra mostraremos as t\'ecnicas usadas para o c\'alculo de uma apresenta\c c\~ao para o grupo $\nu(G)$ e suas respectivas sessões como o $G\otimes G$, $G\wedge G$, $\Delta(G)$, $M(G)$. 
	
	\vspace*{0.5cm} \noindent Este trabalho foi realizado em parceria com Nora\'i Romeu Rocco (UnB).
	
	\vspace{24pt}
%%====================================================================
%	\talktitle[Kostiantyn Iusenko]{Quivers, Algebras and Adjoint functors}
%
%	\authorinfo[iusenko@ime.usp.br]{Kostiantyn Iusenko}{Universidade de S\~{a}o Paulo (IME-USP)}
%
%	\noindent\textbf{Abstract}.\index{Kostiantyn Iusenko}\label{koiu} The aim of this course is a primer with basic notions of category theory and representation theory of finite dimensional algebras. We will study the concept of adjoint functors and show that the correspondence “quiver” $\rightleftarrows$ “algebra” can be interpreted as a pair of adjoint functors between certain categories. Mostly the lectures do not contain the proofs and the theory is accompanied by examples (sometimes introduced as the exercises).  
%
%	
%\vspace{24pt}

%====================================================================
	
	\talktitle[Paulo Henrique de Azevedo Rodrigues]{Zeros p-ádicos de Formas Aditivas}
	
	\authorinfo[paulo.mat.ufg@gmail.com]{Paulo Henrique de Azevedo Rodrigues}{Universidade de Goiás}
	
	\noindent\textbf{Resumo}.\index{Paulo H. A. Rodrigues}\label{phar} 
	Falaremos sobre condições suficientes para que polinômios diagonais tenham zeros não triviais sobre corpos p-ádicos e em alguns casos mostrar que as condições obtidas são as melhores possíveis.
	
	\vspace{24pt}
%====================================================================

	\talktitle[Plamen Koshlukov]{Gradings and graded identities for upper triangular matrices: Lie and Jordan algebras}
	
	\authorinfo[plamen@ime.unicamp.br]{Plamen Koshlukov}{Universidade Estadual de Campinas}
	
	\noindent\textbf{Resumo}.\index{Plamen Koshlukov}\label{pk} 
	We describe the gradings by an arbitrary group, on the Lie and on the
	Jordan algebra of the upper triangular matrices of any order. In the Lie
	case one obtains a reasonably complete description of the corresponding
	graded identities while in the Jordan case such a description seems to be
	quite difficult.
	
	It should be noted that in both the Lie and the Jordan cases, there appear
	some "strange" gradings. Let us recall that in the associative case it is
	known that every group grading is, up to isomorphism, elementary.
	Moreover, the graded identities of such gradings are also well known. Thus
	in the Lie and the Jordan cases there appear sharp differences from the
	associative one.
	
	\vspace{24pt}

%====================================================================
	\talktitle[Victor Gonzalo Lopez Neumann]{Segundo peso mínimo de Hamming de códigos projetivos de Reed-Muller}

	\authorinfo[glopezneumann@gmail.com]{Victor Gonzalo Lopez Neumann}{Universidade Federal de Uberlândia.}

	\noindent\textbf{Resumo}.\index{Victor G. L. Neumann}\label{vgln} 
	Os códigos projetivos de Reed-Muller foram introduzidos por Lachaud em 1988. Sua dimensão e distância mínima foram calculados por Serre e Sorensen em 1991. A distribuição de pesos de Hamming permite estudar a performance de um código, é por isto que o conhecimento do segundo peso mínimo e os seguintes pesos é importante em teoria de códigos. No entanto, até 2017 pouco se sabia sobre os pesos de Hamming para os códigos projetivos de Reed-Muller, inclusive sobre o segundo peso mínimo de Hamming. Nesta palestra, apresentamos as ideias de geometria finita que nos permitiram calcular o segundo peso de Hamming dos códigos projetivos de Reed-Mulller na maior parte dos casos. Trabalhamos no corpo finito com q elementos, onde q é maior ou igual a 2. Em particular, determinamos completamente o segundo peso para q=2 e q=3.

	\vspace*{0.5cm} \noindent Este trabalho foi realizado em parceria com o prof. Cícero Carvalho.

\vspace{24pt}

%====================================================================
%	Fim dos Resumos de Algebra e Teoria dos Numeros
%====================================================================	


\clearpage	