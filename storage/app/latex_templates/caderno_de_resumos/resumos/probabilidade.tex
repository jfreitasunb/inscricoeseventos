%====================================================================
%	Resumos de Probabilidade
%====================================================================

\pagestyle{fancy}

\lhead{}
\chead{{X Workshop de Ver\~{a}o em  Matem\'{a}tica - MAT/UnB}}
\rhead{}
%P\'{a}gina \arabic{page} de \pageref{ultimapagina}

\lfoot{Probabilidade}
\cfoot{\arabic{page}}
\rfoot{Ver\~{a}o \ano}

\renewcommand{\headrulewidth}{0.4pt}
\renewcommand{\footrulewidth}{0.4pt}

%====================================================================
%====================================================================

\begin{center}
	%\vspace{1cm}
	\huge{{\bf Probabilidade}}
	\vspace{1cm}
\end{center}

%====================================================================
%	Inicio dos Resumos de Probabilidade
%====================================================================
	
	\talktitle[Christian Olivera]{2D Navier-Stokes equation with cylindrical fractional Brownian noise}
	
	\authorinfo[colivera@ime.unicamp.br]{Christian Olivera}{Universidade Estadual de Campinas}
	
	\noindent\textbf{Resumo}.\index{Christian Olivera}\label{co} 
	We consider the Navier-Stokes equation on the 2D torus, with a stochastic forcing term  which is a cylindrical fractional Wiener noise of Hurst parameter $H$.We prove local existence and uniqueness when $\frac 38< H<\frac 12$ and global existence and uniqueness when $\frac 12<H<1$.The case $H=\frac12$ has already been considered in the literature.
	
	\vspace{24pt}

%====================================================================
	\talktitle[Eduardo de Amorim Neves]{Brownian Motion and Martingales depending of Metrics and Connections}
	
	\authorinfo[edunev@gmail.com]{Eduardo de Amorim Neves}{Universidade Estadual de Maringá}
	
	\noindent\textbf{Resumo}.\index{Eduardo A. Neves}\label{ean} 
	In this work we will characterize the concept of Brownian motion and Martingale on manifolds that are provided by a family of metrics and connections which depends smoothly on time, after that we will show some Itô formulas for integrations this process. To finish we will show some examples this type of process into surface of revolution.
	
	\nocite{ref10}\nocite{ref11}\nocite{ref12}\nocite{ref13}\nocite{ref14}\nocite{ref15}\nocite{ref16}
		
	\selectlanguage{english}
	
		\begin{thebibliography}{999}
	
		\bibitem{ref10} Neves, E. A; Catuogno, J.P., Diffusion process on manifolds with time-dependent metrics and connections, preprint.
		
		\bibitem{ref11} K.A, Coulibaly., Brownian motion with respect to time-changing riemannian metrics applications to Ricci flow, Ann. Inst. Henri Poincaré Probab. Stat. 47, 515-538, nž2, 2011.
		
		\bibitem{ref12} Guo, H. ; Philipowski, R. ; Thalmaier, A., Martingales on manifolds with time-dependent connection. arxiv:1305.0454, 2013.
		
		\bibitem{ref13} Catuogno, Pedro., Second order connections and stochastic horizontal lifts, Journal of Geometry and Physics, Vol.56, 1637-1653, 2006.
		
		\bibitem{ref14} Catuogno, Pedro., A Geometric Itô Formula, Sociedade Brasileira de Matemática, Vol.33, 85-99, 2007.
		
		\bibitem{ref15} Stelmastchuk, Simão. and Catuogno, Pedro, Martingale on Frame Bundle, Potential Analysis, 61-69, 2008.
		
		\bibitem{ref16} Hsu, Elton P., Stochastic Analysis on Manifolds American Mathematical Society, Vol.38, Rhode Island, 2001.
		
	\end{thebibliography}
	
	\vspace{24pt}

%====================================================================

	\talktitle[Paulo Rufino]{Princípio da Média em Variedades Diferenciáveis}
	
	\authorinfo[paulo.ruffino@gmail.com]{Paulo Rufino}{Universidade Estadual de Campinas}
	
	\noindent\textbf{Resumo}.\index{Paulo Rufino}\label{pr} 
	O termo genérico "princípio da média" se refere a qualquer aproximação assintótica que se faça a partir da média temporal. São portanto resultados ergódicos obtidos em diferentes estruturas geométricas e analíticas. No nosso contexto temos inicialmente um sistema cujas trajetórias permanecem em curvas de nível de uma função diferenciável $F$. Esse sistema é perturbado por um parâmetro $\epsilon \geq 0$ pequeno na direção do gradiente $\nabla F$ desta função. Mostra-se que em relação a uma determinada topologia, a média assintótica da função $F$ ao longo das trajetórias normalizadas por $1/\epsilon$ é solução de uma EDO na reta com derivada dada pela média da norma $\| \nabla F \| $ em cada curva de nível, quando $\epsilon$ tende a zero. Esse contexto se estende do ponto de vista probabilístico para sistemas guiados por movimento Browniano e  processos de Levy.

	
	\vspace{24pt}

%====================================================================
%	Fim dos Resumos de Probabilidade
%====================================================================	


\clearpage	